\subsection{Background}
\begin{frame}

\begin{columns}
    \begin{column}{0.6\textwidth}
        \begin{itemize}
            \item Predicting the neutron flux distribution is crucial for reactor analysis
            \item The flux distribution determines the power distribution, which has important ramifications for design and operation
            \begin{itemize}
                \item Economically, efficient fuel loading patterns and prevention of fuel failures are determined largely by the power distribution
                \item The power distribution also drives safety constraints for both steady-state and transient operation, including accident scenarios
            \end{itemize}
            \item These requirements demand a high degree of accuracy from the codes used in reactor analysis
        \end{itemize}
    \end{column}
\begin{column}{0.4\textwidth}
\vfill
\includegraphics[width=\columnwidth]{reactor-geometry.png}
\vfil
\end{column}
\end{columns}
    
\end{frame}

%%%%%%%%%%%%%%%%%%%%%%%%%%%%%%%%%%%%%%%%%%%%%%%%%%%%%%%%%%%%%%%%%%%%%%%%%%%%%%%%%

\begin{frame}
    
\begin{columns}
    \begin{column}{0.5\textwidth}
        \begin{itemize}
            \item Reactor analysis has traditionally used a two-step approach
            \begin{itemize}
                \item Lattice calculations to generate homogenized cross sections
                \item Nodal diffusion methods to solve global problem with homogenized cross sections
            \end{itemize}
            \item Recent increases in computing power have generated interest in direct, whole-core transport calculations
            \begin{itemize}
                \item Monte Carlo
                \item Deterministic 3D transport
                \item Planar Synthesis Methods: 2D/1D and 2D/3D
            \end{itemize}
        \end{itemize}
    \end{column}
    \begin{column}{0.5\textwidth}
        \vfill
        \includegraphics[width=\columnwidth]{top500-performance.jpg}
        \vfill
    \end{column}
\end{columns}

\end{frame}

%%%%%%%%%%%%%%%%%%%%%%%%%%%%%%%%%%%%%%%%%%%%%%%%%%%%%%%%%%%%%%%%%%%%%%%%%%%%%%%%

\subsection{Motivation}
\begin{frame}
    
    \begin{itemize}
        \item Planar synthesis methods are faster than 3D transport, but still computationally expensive
        \item To make these methods useful practically, runtimes need to be decreased
        \begin{itemize}
            \item Algorithm and methods improvements
            \item Reduction in number of planes
        \end{itemize}
        \item Subgrid methods can be used to maintain accuracy with fewer planes
        \begin{itemize}
            \item Needs to be able to capture local effects of various reactor components
            \item Should be able to be applied to a variety of situations
            \item Cheaper than using more planes
        \end{itemize}
        \item Three new methods developed to accomplish two goals:
        \begin{itemize}
            \item Significant reduction in errors caused by control rod cusping, the most severe axial heterogeneity for planar synthesis methods
            \item Reduce the runtime of the 2D/1D code MPACT for cases with rod cusping
        \end{itemize}
    \end{itemize}

\end{frame}
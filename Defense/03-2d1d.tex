\begin{frame}[t]{Background}

\vspace{-10pt}
\begin{itemize}
  \item 2D/1D method was developed by researchers at Korea Atomic Energy 
  Research Institute (KAERI) \cite{3DHetWholeCoreTransPlanarMOC,DeCARTTheoryManual,MethodsAndPerformanceOfDecart}
  \item Newer 2D/1D code MPACT, jointly developed by University of Michigan and Oak Ridge National Laboratory, is used for this work \cite{MPACTTheoryManual}
\end{itemize} 
\begin{center}
\includegraphics[width=0.7\textwidth]{../figs/2d1d-subplane.png}
\end{center}
  
\end{frame}

%%%%%%%%%%%%%%%%%%%%%%%%%%%%%%%%%%%%%%%%%%%%%%%%%%%%%%%%%%%%%%%%%%%%%%%%%%%%%%%%%

\begin{frame}[t]{Radial Equations}
    
    \begin{itemize}
      \item Average transport equation axially from $z_{k-\frac{1}{2}}$ to 
      $z_{k+\frac{1}{2}}$
      \item Assume cross sections are axially constant in region of integration
    \end{itemize}
    \begin{dmath*}
        {\Omega_x\frac{\partial \psi_{g}^Z}{\partial x} + 
        \Omega_y\frac{\partial \psi_{g}^Z}{\partial y}} + 
        {\Sigma_{tr,g}\left(x,y\right)\psi_{g}^Z\left(x,y,\bm\Omega\right)} = 
        {\textcolor{red}{q_{g}^Z}\left(x,y,\bm \Omega\right)} + 
        {\textcolor{blue}{L_{g}^Z}\left(x,y,\Omega_z\right)}
    \end{dmath*}
    \begin{dmath*}
        {\textcolor{red}{q_{g}^Z}\left(x,y,\bm \Omega\right)} = 
        {\frac{1}{4\pi}\sum_{g'=1}^{G}\intop_{4\pi}\Sigma_{s,g'\rightarrow 
        g}^Z\left(x,y,\bm\Omega'\cdot\bm\Omega\right)\psi_{g'}^Z\left(x,y,\bm\Omega'\right)d\Omega'}
         + {\frac{1}{k_{eff}}\frac{\chi_{g}^Z}{4\pi}\sum_{g'=1}^G\intop_{4\pi} 
        \nu\Sigma_{f,g'}^Z\left(x,y\right)\psi_{g'}^Z\left(x,y,\bm\Omega'\right)d\Omega'}
         + {\frac{Q_{g}^Z\left(x,y\right)}{4\pi}}
    \end{dmath*}
    \begin{align*}
    \textcolor{blue}{L_{g}^Z}\left(x,y,\Omega_z\right) &= \frac{\Omega_z}{\Delta z_k}\left(\psi_{g,z_{k-\frac{1}{2}}} - \psi_{g,z_{k+\frac{1}{2}}}\right) \approx \frac{J_{g,z_{k-\frac{1}{2}}} - J_{g,z_{k+\frac{1}{2}}}}{4\pi\Delta z_k}
    \end{align*}
    
\end{frame}

%%%%%%%%%%%%%%%%%%%%%%%%%%%%%%%%%%%%%%%%%%%%%%%%%%%%%%%%%%%%%%%%%%%%%%%%%%%%%%%%%

\begin{frame}[t]{Axial Equations}

\begin{itemize}
  \item Average transport equation over x from $x_{i-\frac{1}{2}}$ to 
  $x_{i+\frac{1}{2}}$ and over y from $y_{j-\frac{1}{2}}$ to $y_{j+\frac{1}{2}}$
  \item Assume cross sections are radially constant in region of integration
\end{itemize}
\begin{dmath*}
{\Omega_z \frac{\partial \psi_{g}^{XY}}{\partial z}} + 
{\Sigma_{tr,g}^{XY}\left(z\right)\psi_{g}^{XY}\left(z,\bm\Omega\right)} = 
q_{g}^{XY}\left(z,\bm\Omega\right) + 
{\textcolor{blue}{L_{g}^{XY}}\left(z,\Omega_x,\Omega_y\right)}
\end{dmath*}
\begin{equation*}
\textcolor{blue}{L_{g}^{XY}}\left(z,\Omega_x,\Omega_y\right) \approx 
{\frac{J_{g,x_{i-\frac{1}{2}},y_j} - 
  J_{g,x_{i+\frac{1}{2}},y_j}}{4\pi\Delta x_i}} +
\frac{J_{g,x_i,y_{j-\frac{1}{2}}} - J_{g,x_i,y_{j+\frac{1}{2}}}}{4\pi\Delta 
y_j}
\end{equation*}

\end{frame}

%%%%%%%%%%%%%%%%%%%%%%%%%%%%%%%%%%%%%%%%%%%%%%%%%%%%%%%%%%%%%%%%%%%%%%%%%%%%%%%%%

\begin{frame}[t]{Calculation Flow}

\begin{columns}
  \begin{column}{0.6\textwidth}
    \begin{itemize}
      \item 3D Coarse Mesh Finite Difference (CMFD) \cite{SmithCMFDOrig}
      \begin{itemize}
        \item Determines global flux shape to scale fine mesh solution
        \item Calculates radial currents for 1D axial solver
      \end{itemize}
      \item 1D NEM-P$_3$ \cite{SPnEquations,finnemann1977RodCuspingOrigMention}
      \begin{itemize}
        \item Calculates improved axial currents for 2D solver
      \end{itemize}
      \item 2D Method of Characteristics (MOC) \cite{AskerMOCOrig1972,HalsallMOCOrigCACTUS1980}
      \begin{itemize}
        \item Solves for fine mesh scalar flux
        \item Calculates updated radial currents for CMFD calculation
      \end{itemize}
    \end{itemize}
  \end{column}
\begin{column}{0.4\textwidth}
  \begin{figure}[h]
    \centering
    \resizebox{!}{0.7\textheight}{\begin{tikzpicture}[node distance=2cm]

% Begin
\node (start) [startstop] {Start};
\node (init) [io, right of=start, xshift=2.0cm] {Input, Initialize Solution};

% CMFD
\node (homog) [process, right of=init, xshift=2.0cm] {Homogenize CMFD cross-sections};
\node (calcCouplingCoeffs) [process, below of=homog] {Calculate $\tilde{D}$, $\hat{D}$};
\node (3DCMFD) [process, below of=calcCouplingCoeffs] {3D CMFD Calculation};
\node (CMFDupdate) [process, left of=3DCMFD, xshift=-2.5cm] {Update fission source, matrix};
\node (CMFDconv) [decision, below of=3DCMFD, yshift=-1.5cm] {CMFD Flux, k-eff converged?};
\node (proj) [process, below of=CMFDconv, yshift=-1.5cm] {Scale MOC flux with CMFD flux};

% Nodal
\node (radialTL) [process, left of=proj, xshift=-2.5cm] {Calculate radial transverse leakage source};
\node (sp3) [process, left of=radialTL, xshift=-2.0cm] {1D SP$_3$ Calculation};

% MOC
\node (axialTL) [process, below of=sp3,yshift=-1.5cm] {Calculate fission source, Axial transverse leakage source};
\node (MOC) [process, right of=axialTL, xshift=2.0cm] {Perform 2D MOC sweeps};
\node (MOCconv) [decision, right of=MOC,xshift=2.5cm] {Fission source converged?};
\node (output) [io, below of=MOCconv,yshift=-1.5cm] {Output};
\node (stop) [startstop, below of=output] {Stop};

% Basic Arrows
\draw [arrow] (start) -- (init);
\draw [arrow] (init) -- (homog);
\draw [arrow] (homog) -- (calcCouplingCoeffs);
\draw [arrow] (calcCouplingCoeffs) -| (CMFDupdate);
\draw [arrow] (CMFDupdate) -- (3DCMFD);
\draw [arrow] (3DCMFD) -- (CMFDconv);
\draw [arrow] (CMFDconv) -- node[anchor=west] {yes} (proj);
\draw [arrow] (proj) -- (radialTL);
\draw [arrow] (radialTL) -- (sp3);
\draw [arrow] (sp3) -- (axialTL);
\draw [arrow] (axialTL) -- (MOC);
\draw [arrow] (MOC) -- (MOCconv);
\draw [arrow] (MOCconv) -- node[anchor=west] {yes} (output);
\draw [arrow] (output) -- (stop);

% Fancy Arrows
\draw [arrow] (CMFDconv) -| node[anchor=north] {no} (CMFDupdate);
\draw [arrow] (MOCconv) -| node[anchor=north] {no} ([xshift=0.5cm]CMFDconv.east) |- (homog);

\end{tikzpicture}}
  \end{figure}
\end{column}
\end{columns}

\end{frame}

%%%%%%%%%%%%%%%%%%%%%%%%%%%%%%%%%%%%%%%%%%%%%%%%%%%%%%%%%%%%%%%%%%%%%%%%%%%%%%%%%

\begin{frame}[t]{3D CMFD}
    
        \begin{itemize}
          \item Diffusion-based acceleration performed on coarse mesh
          \item $\hat{D}$ coupling coefficients enforce consistency between diffusion and transport solutions
          \begin{equation*}\scriptstyle
          \hat{D}_{g,s} = \frac{J_{g,s}^{trans,k-1} + 
            \tilde{D}_{g,s}\left(\phi_{g,p}^{diff,k} - 
            \phi_{g,m}^{diff,k}\right)}{\left(\phi_{g,p}^{trans,k} + 
            \phi_{g,m}^{diff,k}\right)}
          \end{equation*}
          \item Coarse mesh solution projected onto the fine mesh, preserving MOC radial shape and CMFD volume-averaged flux
          \begin{equation*}\scriptstyle
          \phi_{g,j}^{trans,k} = \frac{\phi_{g,i}^{diff,k}}{\phi_{g,i}^{diff,k-1}} \phi_{g,j}^{trans,k-1}
          \end{equation*}
          \item Subplane scheme is used to capture subplane axial flux shapes
        \end{itemize}
    
\end{frame}

%%%%%%%%%%%%%%%%%%%%%%%%%%%%%%%%%%%%%%%%%%%%%%%%%%%%%%%%%%%%%%%%%%%%%%%%%%%%%%%%%

\begin{frame}[t]{1D NEM-P$_3$}
    
\begin{itemize}
    \item P$_3$ \cite{SPnEquations} used to handle angular shape 
    \begin{itemize}
        \item Angular flux expanded in terms of Legendre polynomials:
            \begin{equation*}\scriptstyle
            \varphi\left(x,\mu\right) \approxeq \sum_{n=0}^N \frac{2n+1}{2}\varphi_n\left(x\right) P_n\left(\mu\right)
            \end{equation*}
            \begin{equation*}\scriptstyle
            \Sigma_s\left(x,\mu,\mu'\right) = \sum_{n=0}^N \frac{2n+1}{2} P_n\left(\mu\right) P_n\left(\mu'\right)\Sigma_{s,n}\left(x\right)
            \end{equation*}
        \item Two diffusion-like equations
            \begin{equation*}\scriptstyle
            {-\bm{\nabla} \cdot D_{0,g} \left(\bm x\right) \bm \nabla 
            \Phi_{0,g}\left(\bm x\right) + \left[\Sigma_{tr,g}\left(\bm 
            x\right) - \Sigma_{s0,g}\left(\bm 
            x\right)\right]\Phi_{0,g}\left(\bm x\right)} = {Q_g\left(\bm 
            x\right) + 2\left[\Sigma_{tr,g}\left(\bm x\right) - 
            \Sigma_{s0,g}\left(\bm x\right)\right]\Phi_{2,g}\left(\bm 
            x\right)}
            \end{equation*}
            \begin{dmath*}\scriptstyle
            {-\bm{\nabla} \cdot D_{2,g} \left(\bm x\right) \bm \nabla 
            \Phi_{2,g}\left(\bm x\right) + \left[\Sigma_{tr,g}\left(\bm 
            x\right) - \Sigma_{s2,g}\left(\bm 
            x\right)\right]\Phi_{2,g}\left(\bm x\right)} = 
            {\frac{2}{5}\left\lbrace \left[\Sigma_{tr,g}\left(\bm x\right) - 
            \Sigma_{s0,g}\left(\bm x\right)\right]\left[\Phi_{0,g}\left(\bm 
            x\right) - 2\Phi_{2,g}\left(\bm x\right)\right] - Q_g\left(\bm 
            x\right) \right\rbrace}
            \end{dmath*}
        \item Iterating between these equations gives solution for $\Phi_0$ and $\Phi_2$, which can be used to solve for $\varphi_0$ and $\varphi_2$:
            \begin{equation*}\scriptstyle
            \varphi_0\left(x\right)= \Phi_0\left(x\right)- 2\Phi_2\left(x\right)\ , \quad \varphi_2\left(x\right) = \Phi_2\left(x\right)
            \end{equation*}
    \end{itemize}
\end{itemize}
  
\end{frame}

%%%%%%%%%%%%%%%%%%%%%%%%%%%%%%%%%%%%%%%%%%%%%%%%%%%%%%%%%%%%%%%%%%%%%%%%%%%%%%%%

\begin{frame}[t]{1D NEM-P$_3$}

        \begin{itemize}
          \item The Nodal Expansion Method (NEM) \cite{finnemann1977RodCuspingOrigMention} used to handle spatial shape
          \begin{itemize}
              \item Expand source and flux as quadratic and quartic polynomials:
                  \begin{equation*}\scriptstyle
                  Q\left(\xi\right) = \sum_{i=0}^2 q_i P_i\left(\xi\right)\ , \quad 
                  \phi\left(\xi\right) = \sum_{i=0}^4 \phi_i P_i\left(\xi\right)
                  \end{equation*}
              \item 3 moment-balance and 2 continuity equations to solve for 5 flux coefficients:
                  \begin{equation*}\scriptstyle
                  \intop_{-1}^1 P_n\left(\xi\right) 
                  \left(-\frac{D}{h^2}\frac{d^2}{d\xi^2}\phi\left(\xi\right) + \Sigma_r 
                  \phi\left(\xi\right) - Q\left(\xi\right)\right)d\xi = 0,\ n=0,1,2
                  \end{equation*}
                  \begin{equation*}\scriptstyle
                  \phi_L\left(1\right) = \phi_R\left(-1\right)\ , \quad 
                  J_L\left(1\right) = J_R\left(-1\right)
                  \end{equation*}
              \item Applying this method to both P$_3$ equations gives a system of 10 equations for each group and each node
          \end{itemize}
        \end{itemize}
    
\end{frame}

%%%%%%%%%%%%%%%%%%%%%%%%%%%%%%%%%%%%%%%%%%%%%%%%%%%%%%%%%%%%%%%%%%%%%%%%%%%%%%%%%

\begin{frame}[t]{2D MOC}

    \begin{itemize}
      \item Solve along a specific direction $\Omega_n$ to reduce the problem from a PDE to an ODE that can be solved analytically
      \begin{equation*}\scriptstyle
      \frac{\partial \psi_{g,n}}{\partial s} + \Sigma_{t,g}\left(\bm r_0 + 
      s\bm\Omega_n\right)\psi_{g,n}\left(\bm r_0 + s\bm\Omega_n\right) = 
      q_{g,n}\left(\bm r_0 + s\bm\Omega_n\right)
      \end{equation*}
      \begin{dmath*}\scriptstyle
        {\psi_{g,n}\left(\bm r_0 + s\bm\Omega_n\right) = \psi_{g,n}\left(\bm 
        r_0\right)\exp{\left(-\intop_0^s \Sigma_{t,g}\left(\bm r_0 + 
        s'\bm\Omega_n\right)ds'\right)}} + {\intop_0^s q_{g,n}\left(\bm r_0 + 
        s'\bm\Omega_n\right)\exp{\left(-\intop_0^{s'} \Sigma_{t,g}\left(\bm r_0 
        + s''\bm\Omega_n\right)ds''\right)}ds'}
      \end{dmath*}
  \item Assume flat source, cross section along track with 
length $L_j$ and spacing $\delta x$
\begin{align*}\scriptstyle
\psi^{out}_{g,i,n,j} &\scriptstyle = \psi^{in}_{g,i,n,j}e^{-\Sigma_{t,g,i} 
    L_j} + \frac{q_{g,i,n}}{\Sigma_{t,g,i}}\left(1 - 
e^{-\Sigma_{t,g,i}L_j}\right) \\\scriptstyle
\overline{\psi}_{g,i,n,j} &\scriptstyle = 
\frac{q_{g,n,i}}{\Sigma_{t,g,i}} + \frac{1 - e^{-\Sigma_{t,g,i} 
        L_j}}{L_j\Sigma_{t,g,i}}\left(\psi^{in}_{g,i,n,j} - 
\frac{q_{g,n,i}}{\Sigma_{t,g,i}}\right) \\\scriptstyle
\overline{\psi}_{g,i,n} &\scriptstyle = \frac{\sum_j 
    \overline{\psi}_{g,i,n,j} \delta x L_j}{\sum_j \delta x L_j}
\end{align*}
    \end{itemize}

\end{frame} 

%%%%%%%%%%%%%%%%%%%%%%%%%%%%%%%%%%%%%%%%%%%%%%%%%%%%%%%%%%%%%%%%%%%%%%%%%%%%%%%%%

\begin{frame}[t]{2D MOC}
  
    \vspace{-10pt}
      \begin{itemize}
        \item Perform ray tracing and store segment information up front
        \item Set up scattering, fission, and axial transverse leakage sources
        \item Solve each long ray one at a time
        \begin{itemize}
            \item Incoming angular flux at each end of long ray is known from boundary conditions
            \item Outgoing angular flux for each segment is used as incoming for subsequent segments
            \item Region-wise scalar flux and surface currents are tallied as each long ray is swept
        \end{itemize}
      \end{itemize}
  \begin{center}
      \includegraphics[width=0.4\textwidth]{moc-raytrace.png}
  \end{center}

\end{frame}
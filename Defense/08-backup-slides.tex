\begin{frame}[t]{Transport-Corrected Scattering Approximation}
    
    \begin{itemize}
        \item Modifies self-scatter and total cross-sections to account for 
        anisotropy while performing isotropic calculations
        \item Neutron Leakage Conservation (NLC) Method: H-1
        \begin{equation*}
        \Sigma_{s0,g\rightarrow g} = \Sigma_{s0,g\rightarrow g} + \frac{1}{3D_g} 
        - \Sigma_{t,g}
        \end{equation*}
        \item In-Scatter Method: B-11, C-12, O-16
        \begin{equation*}
        \Sigma_{s0,g\rightarrow g} = \Sigma_{s0,g\rightarrow g} - 
        \frac{1}{\phi_{1,g}}\sum_{g'=1}^G \Sigma_{s1,g'\rightarrow g}\phi_{1,g'}
        \end{equation*}
        \item Out-Scatter Method: All other isotopes
        \begin{equation*}
        \Sigma_{s0,g\rightarrow g} = \Sigma_{s0,g\rightarrow g} - \sum_{g'=1}^G 
        \Sigma_{s1,g\rightarrow g'}
        \end{equation*}
    \end{itemize}
    
\end{frame}

%%%%%%%%%%%%%%%%%%%%%%%%%%%%%%%%%%%%%%%%%%%%%%%%%%%%%%%%%%%%%%%%%%%%%%%%%%%%%%%%%

\begin{frame}[t]{2D MOC}
    
    \begin{columns}
        \begin{column}{0.6\textwidth}
            \begin{itemize}
                \item Solve along a specific direction $\Omega_n$
                \begin{equation*}\scriptstyle
                \bm r = \bm {r_0} + s \bm \Omega_n \Rightarrow \begin{cases} 
                x\left(s\right) = x_0 + s\Omega_{n,x} \\ y\left(s\right) = y_0 + 
                s\Omega_{n,y} \\ z\left(s\right) = z_0 + s\Omega_{n,z} \end{cases}
                \end{equation*}
                \item Problem reduces from PDE to ODE that can be solved analytically
                \begin{equation*}\scriptstyle
                \frac{\partial \psi_{g,n}}{\partial s} + \Sigma_{t,g}\left(\bm r_0 + 
                s\bm\Omega_n\right)\psi_{g,n}\left(\bm r_0 + s\bm\Omega_n\right) = 
                q_{g,n}\left(\bm r_0 + s\bm\Omega_n\right)
                \end{equation*}
                \begin{dmath*}\scriptstyle
                    {\psi_{g,n}\left(\bm r_0 + s\bm\Omega_n\right) = \psi_{g,n}\left(\bm 
                        r_0\right)\exp{\left(-\intop_0^s \Sigma_{t,g}\left(\bm r_0 + 
                            s'\bm\Omega_n\right)ds'\right)}} + {\intop_0^s q_{g,n}\left(\bm r_0 + 
                        s'\bm\Omega_n\right)\exp{\left(-\intop_0^{s'} \Sigma_{t,g}\left(\bm r_0 
                            + s''\bm\Omega_n\right)ds''\right)}ds'}
                \end{dmath*}
            \end{itemize}
        \end{column}
        \begin{column}{0.4\textwidth}
            \includegraphics[width=\textwidth]{modular_rays.png}
        \end{column}
    \end{columns}
    
\end{frame} 

%%%%%%%%%%%%%%%%%%%%%%%%%%%%%%%%%%%%%%%%%%%%%%%%%%%%%%%%%%%%%%%%%%%%%%%%%%%%%%%%%

\begin{frame}[t]{2D MOC}
    
    \begin{itemize}
        \item Assume flat source, cross-section along track with 
        length $L_j$ and spacing $\delta x$
        \begin{columns}
            \begin{column}{0.6\textwidth}
                \begin{align*}\scriptstyle
                \psi^{out}_{g,i,n,j} &\scriptstyle = \psi^{in}_{g,i,n,j}e^{-\Sigma_{t,g,i} 
                    L_j} 
                \\\scriptstyle
                &\scriptstyle + \frac{q_{g,i,n}}{\Sigma_{t,g,i}}\left(1 - 
                e^{-\Sigma_{t,g,i}L_j}\right) \\\scriptstyle
                \overline{\psi}_{g,i,n,j} &\scriptstyle = 
                \frac{q_{g,n,i}}{\Sigma_{t,g,i}} 
                \\\scriptstyle
                &\scriptstyle + \frac{1 - e^{-\Sigma_{t,g,i} 
                        L_j}}{L_j\Sigma_{t,g,i}}\left(\psi^{in}_{g,i,n,j} - 
                \frac{q_{g,n,i}}{\Sigma_{t,g,i}}\right) \\\scriptstyle
                \overline{\psi}_{g,i,n} &\scriptstyle = \frac{\sum_j 
                    \overline{\psi}_{g,i,n,j} \delta x L_j}{\sum_j \delta x L_j}
                \end{align*}
            \end{column}
            \begin{column}{0.4\textwidth}
                \includegraphics[width=\textwidth]{modular_rays.png}
            \end{column}
        \end{columns}
        \item Modular ray tracing can be used to minimize storage requirements by 
        tracing only portions of problem geometry
    \end{itemize}
    
\end{frame} 

%%%%%%%%%%%%%%%%%%%%%%%%%%%%%%%%%%%%%%%%%%%%%%%%%%%%%%%%%%%%%%%%%%%%%%%%%%%%%%%%%

\begin{frame}[t]{2D MOC}
    
    \begin{columns}
        \begin{column}{0.55\textwidth}
            \begin{itemize}
                \item Perform ray tracing and store segment information up front
                \item Set up scattering, fission, and axial transverse leakage sources
                \begin{itemize}
                    \item Multi-group sweeping
                    \item 1-group sweeping
                \end{itemize}
                \item Parallel Decomposition
                \begin{itemize}
                    \item Spatial (Planar and Radial)- MPI
                    \item Angle - MPI
                    \item Ray - OpenMP
                \end{itemize}
            \end{itemize}
        \end{column}
        \begin{column}{0.45\textwidth}
            \begin{figure}[h]
                \centering
                \resizebox{!}{0.7\textheight}{\begin{tikzpicture}[node distance=2cm]

% Begin
\node (start) [startstop] {Start};
\node (init) [io, right of=start, xshift=2.5cm] {Input N$_{inners}$};

% MOC
\node (begin) [process, below of=init] {Set $n=0$};
\node (source) [process, below of=begin] {Calculate fission and axial transverse leakage sources};
\node (scatSource) [process, below of=source] {Calculate scattering source};
\node (MOC) [process, below of=scatSource] {2D MOC sweep for each plane};
\node (MOCdone) [decision, below of=MOC, yshift=-1.5cm] {$n = N_{inners}$?};z
\node (stop) [startstop, right of=MOCdone, xshift=2.5cm] {Stop};

% Basic Arrows
\draw [arrow] (start) -- (init);
\draw [arrow] (init) -- (begin);
\draw [arrow] (begin) -- (source);
\draw [arrow] (source) -- (scatSource);
\draw [arrow] (scatSource) -- (MOC);
\draw [arrow] (MOC) -- (MOCdone);

% Fancy Arrows
\draw [arrow] (MOCdone) -| node[anchor=north] {no} ([xshift=-1.5cm]MOC.west) |- (scatSource);
\draw [arrow] (MOCdone) -- node[anchor=north] {yes} (stop);

\end{tikzpicture}}
            \end{figure}
        \end{column}
    \end{columns}
    
\end{frame}

%%%%%%%%%%%%%%%%%%%%%%%%%%%%%%%%%%%%%%%%%%%%%%%%%%%%%%%%%%%%%%%%%%%%%%%%%%%%%%%%%

\begin{frame}[t]{2D/1D Decusping Methods}
    
    \begin{itemize}
        \item Neighbor Spectral Index Method - CRX-2K 
        \cite{cho2015CRX2d1dFusionDecusping}
        \begin{itemize}
            \item Spectral index is defined as the ratio of the fast flux to the 
            thermal flux
            \item Spectral index is used in top and bottom neighbor nodes to 
            estimate partially rodded node flux profile
            \item This estimate is used to update cross sections 
            each iteration
        \end{itemize}
        \item nTRACER Method \cite{ICAPPcontrolRodDecuspingNTRACER}
        \begin{itemize}
            \item Solves local problem to generate CMFD constants
            \item Performs CMFD calculations on fine mesh to obtain axial flux 
            profiles
            \item Uses axial flux profiles during full core calculation to 
            homogenize cross sections
        \end{itemize}
        \item Approximate Flux Weighting Method \cite{gehinThesis1992quasi}
        \begin{itemize}
            \item Originally developed for nodal methods, but also implemented in nTRACER \cite{Ryu2017nTRACERWholeCoreTransportSolutionstoC5G7-TDBenchmark}
            \item Assumes that in partially rodded node, rodded flux is similar to node above and unrodded flux is similar to node below
            \item Assumption allows the partially rodded node cross section to be updated easily during iteration
        \end{itemize}
    \end{itemize}
    
\end{frame}

%%%%%%%%%%%%%%%%%%%%%%%%%%%%%%%%%%%%%%%%%%%%%%%%%%%%%%%%%%%%%%%%%%%%%%%%%%%%%%%%%
\begin{itemize}
    \item Importance of predicting flux distributions (safety, economics)
    \item Historical reactor analysis methods
    \item Shift toward direct whole-core transport with increasing computing power
    \item Briefly mention monte carlo vs. deterministic
\end{itemize}

\section{Motivation}

\begin{itemize}
    \item Planar Synthesis Methods
    \item Need for Speed for practical calculations
    \item Axial Heterogeneities
\end{itemize}

\section{Previous Work in 2D/1D}

\begin{itemize}
    \item Introduce rod cusping as most significant heterogeneity
    \item Briefly discuss prior methods used to address rod cusping (but not much since chapter 4 does that in more detail)
    \item Limitations of these methods to motivate new ones -> motivation for better methods that address rod cusping more accurately and other heterogeneities in general
\end{itemize}

\section{Dissertation Overview}

\begin{itemize}
    \item Introduction to transport theory, including standard/relevant approximations to the equation \& numerical methods to solve it
    \item Description of the 2D/1D Method.  Introduce MPACT here
    \item History of prior rod cusping methods, both nodal and whole-core transport
    \item Description of methods I've developed
    \item Results \& Analysis
    \item Conclusions \& Future Work
\end{itemize}
For several decades, it has been typical for a two-step approach to be used in reactor analysis.  This approach consisted of high-fidelity calculations on small regions which were used to inform the 3D nodal calculations for the full core problem.  This allowed for efficient calculations on limited computing resources.  However, the drawback to these methods is that the approximations required to make them fast limited their accuracy.  With computing power increasing rapidly in recent years, interest has shifted toward more direct methods for reactor analysis.  Performing direct 3D transport calculations is still too burdensome for practical calculations, so other methods have been developed to bridge the gap between the older nodal calculations and direct 3D calculations.  One of the more popular methods to do this is the 2D/1D method, which takes advantage of reactor geometry to perform very detailed calculations in the radial direction, then couple those calculations with faster, approximate calculations in the axial direction.  This calculation scheme provides much greater detail than the older nodal methods were with less computational burden than direct 3D calculations.

One chronic problem for nodal methods was that of rod cusping.  This effect arises from the assumption that each node in the core was axially homogeneous.  Reactor control rods may be placed in locations that violate that assumption, requiring the rodded and unrodded materials to be homogenized together.  A simple volume homogenization is the simplest way to handle this, but introduces large errors by increasing the amount of absorption in the node.  While the 2D/1D method may significantly improve on nodal methods, it still requires homogeneity in the axial direction for each 2D plane.  Because of this, rods which are partially inserted into a plane can introduce rod cusping effects when using 2D/1D as with nodal methods.   Many different methods have been proposed to address this problem in both nodal methods and 2D/1D.  Some of the methods are fast and simple to implement, but only reduce the errors introduced by rod cusping.  Others do an excellent job of eliminating the errors, but require additional detailed calculation, increasing the computational expense of the solution.  The work presented here seeks to resolve the rod cusping problem in the 2D/1D code MPACT through new methods which are both fast and accurate.

First, the neutron transport equation will be introduced, along with some of the most important approximations commonly used to solve it.  Several important numerical methods will then be introduced, with a focus on those used to solve the rod cusping problem in this work.  This will be followed by a more detailed history and description of 2D/1D scheme, which will lead into a discussion of the various control rod decusping methods which have been developed in the past.  With this foundation laid, the recent methods development in MPACT will be described, and the results of these methods will be presented.  Finally, a new transport-based decusping method will be proposed which builds on the work presented here to address the rod cusping problem in a manner that is efficient and physically consistent with the 2D-1D method.
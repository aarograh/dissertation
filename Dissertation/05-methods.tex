This chapter will describe three control rod decusping methods developed as part of this work.  The first method is polynomial decusping, a simple correction based on pregenerated data.  The second method is subplane collision probabilities.  This method extends the subplane scheme by introducing a 1D collision probabilities calculation to capture subgrid information around partially inserted rods.  The final method presented here is the subray method of characteristics.  This modified form of MOC directly accounts for the partially inserted rod in the 2D MOC calculations instead of applying corrections to cross sections afterwards.

\section{Polynomial Decusping}

The polynomial decusping technique was developed to provide a fast, simple correction to the cusping problem in the MPACT code.  This technique assumes that the reactivity and power around the partially inserted rod have a predictable shape as functions of the rod position within the MOC plane.  Based on this assumption, correction factors were developed to reduce the volume fraction of control rod material and reduce the magnitude of the cusping effects.

To do this, a 3$\times$3 assembly problem with a control rod in the center assembly was used (this problem is described in more detail in Section \hl{number}).  

\section{Subplane Collision Probabilities}

The subplane scheme as it was originally conceived was used primarily to address stability issues caused by thin MOC planes in early 2D/1D codes.  While other improvements to the 2D/1D method have largely eliminated these problems, the idea of capturing subplane information using the subplane scheme can be useful for addressing partially inserted rods without substantially increasing the computational cost.  To do this, three changes were made to the basic subplane scheme: an axial correction, a radial correction, and an MOC cross section correction.

\subsection{Axial Correction}

Traditionally, the subplane scheme uses axially constant cross sections for all subplanes in each MOC plane.  When a control rod is partially inserted in the plane, a flux-volume homogenized cross section is calculated and used for each subplane.  This allows the subplane scheme to be used, but does little to account for the partially inserted rod.

To resolve this issue, the subplanes with the control rod use the rod cross section, and the subplane without the rod use the moderator cross section.  These cross sections are still homogenized using flux-volume weighting as before, but without homogenizing axially.  Doing this allows both the CMFD and P$_3$ calculations to capture some of axial effects of the partially inserted rod, reducing the magnitude of the cusping errors around the rod.

\subsection{Radial Correction}

Using axially heterogeneous cross sections within an MOC plane corrects some of the cusping effects, but it does not accurately capture the radial effects of the partially inserted rod.  In reality, the radial flux shape in the rodded subplanes is completely different from the shape in the unrodded subplanes.  The MOC calculations are done on the thicker MOC planes using axially homogenized cross sections in the partially rodded regions.  This produces a radial flux shape that is not representative of either the rodded or unrodded region, as illustrated in figure \hl{number}.  Thus, using this radial flux shape to calculate the pin-homogenized cross sections for CMFD P$_3$ introduces some error in the cross sections.

\begin{figure}
    \centering
    \caption{Radial flux shapes for rodded, unrodded, and axially homogenized MOC pin cells}\label{f:radialFluxProfiles}
\end{figure}

To account for this, a 1D collision probabilities calculation is introduced during the homogenization step.  These calculations are done in the rodded and unrodded regions to generate separate radial flux shapes that can be used during the CMFD homogenization step, as shown in Figure \ref{f:SCPdecusping}.  These CP calculations are done for each energy group and consist of inverting a small $N_r \times N_r$ matrix, where $N_r$ is the number of rings in the pin cell model (usually less than 10).  Because these matrices are so small, these calculations are computationally negligible when compared with the full 2D/1D iteration scheme while providing more accurate radial information for the cross section homogenization.

\begin{figure}
    \centering
    \includegraphics[width=0.8\textwidth]{CPdecusp.png}
    \caption{Illustration of subplane collision probabilities used on a partially inserted rod}\label{f:SCPdecusping}
\end{figure}

\subsection{MOC Correction}

The final step in this decusping technique is to use the subplane information from the CMFD/P$_3$ calculations to improve the MOC calculations.  To do this, the volume-homogenized cross sections are updated each iteration using a flux-volume weighting that involves the axial flux shape from the CMFD/P$_3$ system and the radial flux shape from the CP calculations.  These are combined the same was as the nTRACER method using Equation \ref{e:nTRACERdecusping}.

\section{Subray Method of Characteristics}

The previous two methods are each effective in reducing the effects of rod cusping, as will be shown in Chapter \hl{number}.  Despite these improvements, each method has significant drawbacks:
\begin{itemize}[leftmargin=*]
    \item \textbf{Polynomial Decusping}
    
    \begin{itemize}
        \item Corrections are limited to certain control rod materials (AIC, B$_4$C, Tungsten).
        \item Results are worse for problems significantly different from the one described in Section \hl{number}.
    \end{itemize}

    \item \textbf{Subplane Collision Probabilities}
    
    \begin{itemize}
        \item CP is limited to TCP$_0$ scattering.
        \item CP implementation for non-cylindrical geometries can be complicated (e.g., BWR control blades).
        \item Instability can be introduced in the CP calculations by large transport corrections.
        \item MOC calculations are still performed using homogenized cross sections.
    \end{itemize}
\end{itemize}
The most important inaccuracy in the subplane CP method is the final point.  To fully address the partially inserted rod, it is desirable to account for the rod in the 2D MOC calculations themselves using heterogeneous cross sections.  Doing so will greatly improve the overall accuracy of the 2D/1D calculation by improving the accuracy of the transport itself.

\subsection{1D MOC Code and Problem Description}

The code is set up to take in a description of pins and materials to be used for the calculations.  For the geometry, a pin pitch is specified which is used for all pins.  Each pin consists of a list of radii.  Assuming a square pin cell, these pins are then transformed from cylindrical geometry to slab geometry while preserving the volume fraction of each material.  Thus, the thickness of the pin slab is equal to the pin pitch, but the width of the fuel material will not be equal to input radius, since the volume fractions are preserved.  One material is then specified for each region, though each material region is divided into many sub-regions for the MOC calculations.  These materials are defined by a separate cross-section library file.  This file uses the ``user library'' format supported by MPACT, which allows the user to put in macroscopic cross-sections for absorption, nu-fission, kappa-fission, chi, and scattering moments.  For all these calculations, the C5G7 benchmark cross-sections \cite{EELewisC5G72003,EELewisC5G7extended2005} were used.  These cross sections are included in Appendix \ref{app:c5g7xs}.

For the MOC sweeps, a Gaussian quadrature \cite{HandbookOfMathFunctions1972} is used with 2, 4, 8, 16, or 32 polar angles, with half of the angles being used in each direction.  The MOC sweeps are done similarly to how they are done in MPACT, with the loop over energy groups being the innermost loop.  The code can be run as either a fixed source solver or an eigenvalue solver.  For the eigenvalue mode, power iteration is used after each MOC calculation to determine and updated k$_{eff}$.  The fixed source mode can be used to run either a specified number of iterations or to run until the scattering source is converged below some tolerance.  This allows some flexibility on exactly what kinds of results can be obtained.

The problem used for these calculations was a 1D variation of VERA Problem 4.  The center row of pins across all three assemblies was pulled out and used for the 1D model, resulting in a row of 3$\times$17 pins with a pin pitch of 1.26 cm (the inter-assembly gap was neglected for this model).  The center assembly had 4 guide tubes in it which contained a mixture of moderator and control rod to represent a partially inserted control rod.  These partially rodded locations were the only part of the problem that had any material changes.  This allowed the effects of the cross-section homogenization to be isolated for each calculations.

\subsection{1D MOC Investigation}

\subsubsection{Specified Total Source}

The first set of calculations performed were done using a specified total source.  To do this, the guide tubes were filled with 50\% control rod and 50\% moderator by volume fraction, and a full eigenvalue calculation was performed.  The source distribution from this calculation (both fission and scattering source) were then passed to the the fixed source solver.  A single iteration was run using this source on three different variations of the problem: the 50-50 mixture, fully rodded, and fully unrodded.  Because the multi-group source is set up before performing any MOC sweeps, this resulted in all three of those calculations having an identical source for the MOC sweep.  The only difference between them was the cross-sections used in the guide tubes.

Figure \ref{f:1dmoc-fixed-50-scalflux7} shows the scalar flux resulting from these three calculations.  The most important thing to note in this data is that the effects of the rod are very local in the MOC calculation.  Moving through the rodded pin cell, the rodded, unrodded, and mixed cases have converged back to the same shape by the time the edge of the pin cell is reached.  This indicates that whatever treatment is used for the partially rodded cell likely does not need to worry about interference between neighboring rods because the effects are so localized.

\begin{figure}[H]
    \centering
    \includegraphics[width=0.7\textwidth]{1dmoc-50mix-fixedscat-scalflux7.png}
    \caption{Group 7 scalar flux comparisons for a fixed fission and scattering source calculation}\label{f:1dmoc-fixed-50-scalflux7}
\end{figure}

Figure \ref{f:1dmoc-fixed-50-angflux} shows the right-going angular flux in groups 1 (fast) and group 7 (thermal).  The group 7 angular flux behaves similarly to the group 7 scalar flux in that the effects are localized around each rod.  The three different angular flux shapes are still somewhat different at the edge of the neighboring pin cell, but have pretty much converged upon reaching the clad and fuel.  The reason for this is that the mean free path of thermal neutrons is small.  The total group 7 cross-section in the moderator is about 2.65 cm$^{-1}$, which corresponds to a mean free path of about 0.38 cm, which is less than one third of the pin pitch for a typical PWR.  Because of this, the differences between the rodded and unrodded cases are washed out quickly if the source distribution is kept the same between the two calculations.

\begin{figure}[H]
    \centering
    \subfigure[Group 1]{
        \centering
        \includegraphics[width=0.6\textwidth]{1dmoc-50mix-fixedscat-angflux1.png}
        \label{f:1dmoc-fixed-50-angflux1}
    }
    ~
    \subfigure[Group 7]{
        \centering
        \includegraphics[width=0.6\textwidth]{1dmoc-50mix-fixedscat-angflux7.png}
        \label{f:1dmoc-fixed-50-angflux7}
    }
    \caption{Angular flux comparisons for a fixed fission and scattering source calculation}\label{f:1dmoc-fixed-50-angflux}
\end{figure}

The same cannot be said for the fast flux.  The mean free path of the fast flux is about 6.3 cm, which is the width of five pin cells.  Thus, it can be seen that each of the control rods after the first builds on the effects of the previous control rod.  While the fast flux does not have a significant impact on the fission source distribution, it does impact the scattering source distribution, which is not shown in these results.

\subsubsection{Fixed Fission Source}

\begin{figure}[H]
    \centering
    \subfigure[25\% Mixture]{
        \centering
        \includegraphics[width=0.45\textwidth]{1dmoc-25mix-angflux7.png}
        \label{f:1dmoc-25-angflux7}
    }
    \hfill
    \subfigure[50\% Mixture]{
        \centering
        \includegraphics[width=0.45\textwidth]{1dmoc-50mix-angflux7.png}
        \label{f:1dmoc-50-angflux7}
    }
    ~
    \subfigure[75\% Mixture]{
        \centering
        \includegraphics[width=0.45\textwidth]{1dmoc-75mix-angflux7.png}
        \label{f:1dmoc-75-angflux7}
    }
    \caption{Group 7 angular flux comparisons for 25\% and 75\% mixtures}\label{f:1dmoc-angflux7}
\end{figure}

The second set of calculations that was performed used a fixed fission source, but allowed the scattering source to fully converge for each calculation.  As in the previous section, an eigenvalue calculation was completed using partially rodded cross-sections.  This was done for 25\%, 50\%, and 75\% rodded cases.  For each case, a fixed fission source calculation was done with the fully rodded and fully unrodded cross-sections.  This time, multiple iterations were allowed for each material to converge the scattering source.  This allows us to see the effects of the rod on the scattering source distribution without worrying about changes in the eigenvalue and fission source distribution.

\begin{figure}[H]
    \centering
    \subfigure[25\% Mixture]{
        \centering
        \includegraphics[width=0.45\textwidth]{1dmoc-25mix-angflux1.png}
        \label{f:1dmoc-25-angflux1}
    }
    \hfill
    \subfigure[50\% Mixture]{
        \centering
        \includegraphics[width=0.45\textwidth]{1dmoc-50mix-angflux1.png}
        \label{f:1dmoc-50-angflux1}
    }
    ~
    \subfigure[75\% Mixture]{
        \centering
        \includegraphics[width=0.45\textwidth]{1dmoc-75mix-angflux1.png}
        \label{f:1dmoc-75-angflux1}
    }
    \caption{Group 1 angular flux comparisons for 25\% and 75\% mixtures}\label{f:1dmoc-angflux1}
\end{figure}


Figure \ref{f:1dmoc-angflux7} shows the group 7 angular flux comparisons for each of the three mixtures.  We immediately see that for each of them, the angular flux for the mixture is closer to the rodded result than the unrodded result than what might be expected based on the volume fraction of the rod.  For example, comparing Figures \ref{f:1dmoc-50-angflux7} and \ref{f:1dmoc-fixed-50-angflux7} shows that the angular flux is much closer to the rodded solution than when the scattering source was fixed, despite the small mean free path of thermal neutrons.  Figure \ref{f:1dmoc-angflux1} shows the same comparisons for the group 1 angular flux.  While the difference in magnitude between each of the three cases is small for any mixture, the long mean free path means that these differences get spread out over a larger area.  This changes the shape of the scattering source for the thermal groups, which then has a more widespread effect on the the scalar flux distribution.  It should be noted as well that all angular flux plots shown here are for the flattest polar angle.  Since the steeper angles travel through more material in each region, the differences between them go away more quickly.  Thus, the angle shown in these plots is the one having the largest impact on the solution.

\subsection{Subray MOC Description}

\begin{figure}
    \centering
    \includegraphics[width=0.8\textwidth]{sub-ray_illustration.png}
    \caption{Illustration of subray method of characteristics for partially inserted rod}\label{f:subrayMOC}
\end{figure}
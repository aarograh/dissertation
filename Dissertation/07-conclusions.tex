The theory and history of rod cusping and decusping methods were presented and used to motivate development of new methods which improved on some aspects of previous methods.  These methods were tested using VERA Problem 4, with the results showing significant improvement over the previous decusping method in MPACT.  Furthermore, it was shown that the decusping methods introduced negligible runtime increases.

Despite the improved results using the new decusping methods, there is still much room for improvement.  Some of this improvement would be found by addressing some of the fundamental approximations of the 2D/1D method.  One example of this would be through addressing the spatial and angular shapes of the axial and radial transverse leakages.  This could prove especially important around a control rod tip where there are strong gradients in the shape of the flux.  Other improvements relate more to the decusping methods themselves.  Increasing the number of dimensions of the 1D CP solver or replacing it with a different method such as the MOC would also likely improve the results of the decusping calculations.  Making these improvements should result in a significant reduction in the remaining error caused by partially inserted rods.

Lastly, a new method was introduced to address the rod cusping problem.  Previous methods focus on correcting the results of a calculation and improving homogenized cross-sections, but the sub-ray MOC method would improve on this by directly accounting for the partially inserted rod when calculating the angular flux, the most fundamental quantity calculated in the 2D/1D method.  Doing this should greatly improve the results of the calculations, while also generalizing the decusping method's applicability to any geometry without significant increases in runtime.

To complete this work, the plan laid out in Table \ref{t:subrayExecutionPlan} will be followed.  This consists of developing the new sub-ray MOC method and prototyping it in a 1D MOC code to work out the details of the MOC implementation.  From there, the method will be implemented in MPACT, using the sub-plane scheme to provide the necessary source information.  The work will be considered complete after a successful implementation of the method and thorough testing to show the effectiveness of the method.  The cases chosen for this testing are VERA Progression Problems 4 (3x3 assembly), Problem 5 (steady-state full core), and Problem 9 (cycle depletion with feedback).  Additionally, a rod ejection case will also be tested to show the method's effectiveness for transient problems in addition to steady-state and slowly changing depletion problems.
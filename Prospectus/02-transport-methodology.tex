\section{Boltzmann Equation}

The Boltzmann equation for neutron transport is shown below: \ifdraft{\todo{Some kind of background or derivation of this?}}{}

\begin{dmath}\label{e:boltzmann}
{\frac{1}{v} \frac{\partial \psi}{\partial t} + \bm \Omega \cdot \bm \nabla \psi + \Sigma_t\left(x,E,t\right)\psi\left(x,E,\bm \Omega,t\right)} = {\intop_0^{\infty} \intop_{4\pi} \Sigma_s\left(x,E' \rightarrow E, \bm \Omega' \cdot \bm \Omega\right) \psi\left(x,E',\bm \Omega'\right) d\bm \Omega' dE'} + {\frac{\chi\left(E\right)}{4\pi} \intop_0^{\infty} \intop_{4\pi} \nu \Sigma_f\left(x,E,t\right) \psi\left(x,E,\bm \Omega,t\right) d\bm\Omega dE + Q\left(x,E,\bm \Omega,t\right)}
\end{dmath}

For many problems, the time dependence of the Boltzmann equation can be neglected.  Doing do eliminates the first term of the equation above.  In order to ensure neutron balance, this equation is normally formulated as an eigenvalue problem.  The eigenvalue $\lambda = \frac{1}{k}$ allows the two sides of the equation to balance.  The cross-sections can then be adjusted until $\lambda = 1$ is achieved, which is the only meaningful solution for steady-state problems.  The eigenvalue form of the Boltzmann equation is shown below:

\begin{dmath}\label{e:ssboltzmann}
{\bm \Omega \cdot \bm \nabla \psi + \Sigma_t\left(x,E\right)\psi\left(x,E,\bm \Omega\right) = \intop_0^{\infty} \intop_{4\pi} \Sigma_s\left(x,E' \rightarrow E, \bm \Omega' \cdot \bm \Omega\right) \psi\left(x,E',\bm \Omega'\right) d\bm \Omega' dE'} + {\frac{\chi\left(E\right)}{4\pi} \intop_0^{\infty} \intop_{4\pi} \nu \Sigma_f\left(x,E,t\right) \psi\left(x,E,\bm \Omega,t\right) d\bm\Omega dE + Q\left(x,E,\bm \Omega\right)}
\end{dmath}

Before addressing the methods used to solve equation \ref{e:ssboltzmann}, we will briefly define each of the terms in the equation.  The first term (equation \ref{e:BTEtermsStreaming} is the streaming term.  This term is made up of the scalar product of the flight direction $\bm \Omega$ and the gradient of the angular flux $\bm \nabla \psi$.  This describes the loss of neutrons from each point in space due to their movement to other points in space.

\begin{equation}\label{e:BTEtermsStreaming}
\bm \Omega \cdot \bm \nabla \psi
\end{equation}

The second term is the total reaction rate (equation \ref{e:BTEtermsTotalRR}).  The angular flux $\psi\left(x,E\bm\Omega\right)$ multiplied by the total macroscopic cross-section $\Sigma_t\left(x,E\right)$ gives the total reaction rate for neutrons in $\left(x,E,\bm\Omega\right)$.  This, together with the streaming term, gives the total loss of neutrons for a steady-state problem.

\begin{equation}\label{e:BTEtermsTotalRR}
\Sigma_t\left(x,E\right)\psi\left(x,E\bm\Omega\right)
\end{equation}

The first term on the right-hand side is the scattering source of neutrons from all energies and angles $E'$ and $\bm\Omega'$ into $E$ and $\bm\Omega$, shown in equation \ref{e:BTEtermsScatteringSource}.  Normally the differential scattering cross-section would be written as $\Sigma_s\left(x,E'\rightarrow E,\bm\Omega'\rightarrow\bm\Omega\right)$.  Because the scattering is symmetric about the incoming direction $\bm\Omega'$, the results of the scattering are only dependent on the cosine of the angle between the directions, allowing the cross-section to be written as a function of the dot product between the angles instead.

\begin{equation}\label{e:BTEtermsScatteringSource}
\intop_0^{\infty} \intop_{4\pi} \Sigma_s\left(x,E' \rightarrow E, \bm \Omega' \cdot \bm \Omega\right) \psi\left(x,E',\bm \Omega'\right) d\bm \Omega' dE'
\end{equation}

The second term on the right-hand side is the source of neutrons from fission, shown in equation \ref{e:BTEtermsFissionSource}.  The integrals over energy and angle gives the total fission source, while the distribution $\chi\left(E\right)$ gives the amount of the fission source that is deposited into energy $E$.  Fission is isotropic, requiring the total fission source to be divided by $4\pi$.  Finally, in order to ensure balance between the left- and right-hand sides of the equation, the fission source is usually multiplied by the eigenvalue $\frac{1}{k}$.  This allows for a mathematical solution to the equation even if the loss and production terms are not equal.  When these terms are equal, the eigenvalue is 1, which means that a solution to the steady-state equation is physically possible.

\begin{equation}\label{e:BTEtermsFissionSource}
\frac{\chi\left(E\right)}{4\pi} \intop_0^{\infty} \intop_{4\pi} \nu \Sigma_f\left(x,E,t\right) \psi\left(x,E,\bm \Omega,t\right) d\bm\Omega dE
\end{equation}

Equation \ref{e:BTEtermsExtSource} shows the final term in the transport equation.  This term is the external source.  This includes all other neutrons which are emitted from means other than in-scatter and fission.

\begin{equation}\label{e:BTEtermsExtSource}
Q\left(x,E,\bm\Omega\right)
\end{equation}

\subsection{Multi-group Approximation}

One important approximation that is commonly made to the transport equation is the multi-group approximation.  To make this approximation, an appropriate energy range for the problem of interest is selected.  This energy range is divided up into $N$ energy groups, with each group going from $E_g$ up to $E_{g-1}$.  For light-water reactor problems, it is common to select 0 eV for $E_N$ and 20 MeV for $E_0$.  Once an appropriate group structure has been selected, equation \ref{e:ssboltzmann} (or \ref{e:boltzmann}) is operated on by 

\begin{equation}
\frac{1}{E_{n-1} - E_n}\intop_{E_n}^{E_{n-1}}\left(\cdot\right)dE
\end{equation}

for each energy group, giving the following result:

\begin{dmath}\label{e:multigroupboltzmann}
{\bm \Omega \cdot \bm \nabla \psi_g + \Sigma_{t,g}\left(x\right)\psi_g\left(x,\bm \Omega\right) = \intop_{4\pi} \Sigma_{s,g'\rightarrow}\left(x,\bm \Omega' \rightarrow \bm \Omega\right) \psi_{g'}\left(x,\bm \Omega'\right) d\bm \Omega'} + {\frac{1}{k}\frac{\chi_g}{4\pi} \sum_{g'=1}^{G} \nu \Sigma_{f,g'}\left(x,t\right) \psi_{g'}\left(x,\bm \Omega\right) + Q_g\left(x,\bm \Omega\right)}
\end{dmath}

In equation \ref{e:multigroupboltzmann}, $\psi_g$ denotes the average angular flux for group $g$, defined as

\begin{equation}\label{e:multigroupangflux}
\psi_g\left(x,\bm \Omega\right)=\frac{1}{E_{n-1}-E_n} \intop_{E_n}^{E_{n-1}} \psi\left(x,E,\bm\Omega\right) dE
\end{equation}

The quantities that we want to preserve through the use of the multi-group approximation are the total reaction rates.  To do this, we must define the multi-group cross-sections as follows:

\begin{equation}\label{e:multigroupXS}
\Sigma_x,g \psi_g = \frac{1}{E_{n-1}-E_n} \intop_{E_n}^{E_{n-1}} \psi\left(E\right) \Sigma_x\left(E\right) dE \Rightarrow \Sigma_x,g = \frac{\intop_{E_n}^{E_{n-1}} \psi\left(E\right) \Sigma_x\left(E\right) dE}{\intop_{E_n}^{E_{n-1}} \psi\left(x,E,\bm\Omega\right) dE}
\end{equation}

This definition of the multi-group flux and multi-group cross-sections guarantees that the total reaction rates in each group will be preserved.

\ifdraft{\hl{Talk about the dependence of the XS on the solution?}}{}

\subsection{Discrete Ordinates Approximation}

To discretize the transport equation in the angular variable, the discrete ordinates (S$_n$) method can be applied.  To apply this method, we integrate the transport equation over a portion of the angular domain:

\begin{dmath}
\intop_{\mu_{i-\frac{1}{2}}}^{\mu_{i+\frac{1}{2}}}\intop_{\gamma_{j-\frac{1}{2}}}^{\gamma_{j+\frac{1}{2}}} \left[ \bm \Omega \cdot \bm \nabla \psi + \Sigma_t\left(x,E\right)\psi\left(x,E,\bm \Omega\right) = \intop_{4\pi} \Sigma_s\left(x,E' \rightarrow E, \bm \Omega' \rightarrow \bm \Omega\right) {\psi\left(x,E',\bm \Omega'\right)} d\bm \Omega' + \frac{1}{k}\frac{\chi\left(E\right)}{4\pi} \intop \nu \Sigma_f\left(x,E\right) \psi\left(x,E,\bm \Omega\right) dE + Q\left(x,E,\bm \Omega\right) \right] d\gamma d\mu
\end{dmath}

\ifdraft{
\begin{itemize}
\item \hl{Make sure scalar flux is defined in terms of angular flux somewhere}
\item \hl{Show discretization}
\item \hl{Mention quadratures}
\end{itemize}
}{}

\subsection{Spherical Harmonics Approximation}

One of the biggest challenges to solving the transport equation is angle dependence of the scattering cross-sections and angular flux.\ifdraft{\hl{Rearrange this: Introduce Legendre polynomials, then show how that simplifies the scattering source/XS}}{}  To simplify the equation while preserving accuracy of the transport equation, the spherical harmonics approximation is applied to the scattering cross-sections.  This approximation makes use of Legendre polynomials, which can be defined via the following recurrence relation:

\begin{subequations}
\begin{equation}
P_{n+1}\left(x\right) = \frac{\left(2n+1\right)xP_n\left(x\right) - nP_{n-1}\left(x\right)}{n+1}
\end{equation}
\begin{equation}
P_0\left(x\right) = 1
\end{equation}
\begin{equation}
P_1\left(x\right) = x
\end{equation}
\end{subequations}

Furthermore, all Legendre polynomials are orthonormal, meaning that they obey the following property:

\begin{subequations}
\begin{equation}
\intop_{-1}^1 P_n\left(x\right) P_m\left(x\right) dx = \frac{2}{2n+1}\delta_{n,m}
\end{equation}
\begin{equation}
\delta_{n,m} = \begin{cases} 1, & n = m \\
0, & n \neq m
\end{cases}
\end{equation}
\end{subequations}

These polynomials can be used to express the differential scattering cross-section as a summation of the Legendre polynomials, as shown in equaiton \ref{e:sphericalHarmonicsXSExpansion}.

\begin{equation}\label{e:sphericalHarmonicsXSExpansion}
\end{equation}

\ifdraft{\hl{Look at Larsen's notes to remember the exact mathematical details of this}}{}

\subsection{Diffusion Approximation}

\ifdraft{\hl{All over this, but need to finish up the Pn stuff first}}{}

\subsection{Transport Correction}

\ifdraft{\hl{Not 100\% sure if this even needs to exist at all}}{}

\section{Numerical Methods}

A wide variety of numerical methods exist to solve the transport equation, but all of them can be categorized in one of two ways: stochastic (Monte Carlo) or deterministic methods.  Monte Carlo methods use random numbers to directly sample probability distribution functions based on the material cross-sections.  These methods can produce highly accurate, detailed results and make no assumptions about the materials or geometry of the problem.  However, because of the random nature of the method, there is always some uncertainty in the answer.  For large problems, the number of particles that must be simulated to achieve acceptable levels of uncertainty is often too burdensome \ifdraft{\todo{Need a better word}}{} for practical calculations.

All the work discussed in the following sections makes use of deterministic methods.  There are many different deterministic methods, but only the ones relevant to this work will be discussed in the coming sections.

\ifdraft{\subsection{Method of Characteristics}

One method that is commonly used to solve the transport equation is the Method of Characteristics (MOC).  This method allows the transport equation to be solved along a characteristic direction without any approximations, making it useful for complicated geometries, such as those found in nuclear reactors.  To derive this method, the right-hand side of equation \ref{e:ssboltzmann} is replaced by the variable $q$:

\begin{subequations}
\begin{equation}
\bm \Omega \cdot \bm \nabla \psi + \Sigma_t\left(x,E\right)\psi\left(x,E,\bm \Omega\right) = q\left(x,E.\bm \Omega\right)
\end{equation}
\begin{dmath}\label{e:MOCqboltzmann}
{q\left(x,E,\bm \Omega\right) = \intop_{4\pi} \Sigma_s\left(x,E' \rightarrow E, \bm \Omega' \rightarrow \bm \Omega\right) \psi\left(x,E',\bm \Omega'\right) d\bm \Omega'} + {\frac{1}{k}\frac{\chi\left(E\right)}{4\pi} \intop \nu \Sigma_f\left(x,E\right) \psi\left(x,E,\bm \Omega\right) dE + Q\left(x,E,\bm \Omega\right)}
\end{dmath}
\end{subequations}

Now the characteristic direction is introduced:

\begin{equation}
\bm r = \bm {r_0} + s \bm \Omega \Rightarrow \begin{cases} x\left(s\right) = x_0 + s\Omega_x \\ y\left(s\right) = y_0 + s\Omega_y \\ z\left(s\right) = z_0 + s\Omega_z\end{cases}
\end{equation}

Substituting this variable into equation \ref{e:MOCqboltzmann} gives the characteristic form of this equation:

\begin{subequations}
\begin{equation}
\frac{\partial \psi}{\partial s} + \Sigma_t\left(\bm r0_0 + s\bm\Omega,E\right)\psi\left(\bm r0_0 + s\bm\Omega,E,\bm \Omega\right) = q\left(\bm r0_0 + s\bm\Omega,E.\bm \Omega\right)
\end{equation}
\begin{dmath}\label{e:characteristicboltzmann}
{q\left(\bm r0_0 + s\bm\Omega,E,\bm \Omega\right) = \intop_{4\pi} \Sigma_s\left(\bm {r_0} + s \bm\Omega',E' \rightarrow E, \bm \Omega' \rightarrow \bm \Omega\right) \psi\left(\bm {r_0} + s\bm\Omega',E',\bm \Omega'\right) d\bm \Omega'} + {\frac{1}{k}\frac{\chi\left(E\right)}{4\pi} \intop \nu \Sigma_f\left(\bm r0_0 + s\bm\Omega,E\right) \psi\left(\bm {r_0} + s\bm\Omega,E,\bm \Omega\right) dE + Q\left(\bm {r_0} + s\bm\Omega,E,\bm \Omega\right)}
\end{dmath}
\end{subequations}

This equation is easily solved using an integrating factor

\begin{equation}
\exp{-\intop_0^s \Sigma_t\left(\bm {r_0} + s'\bm\Omega,E\right)ds'}
\end{equation}

giving the following solution:

\begin{dmath}\label{e:MOCexact}
{\psi\left(\bm {r_0} + s\bm\Omega,\bm\Omega,E\right) = \psi\left(\bm {r_0},\bm\Omega,E\right)\exp{\left(-\intop_0^s \Sigma_t\left(\bm {r_0} + s'\bm\Omega,E\right)ds'\right)}} + {\intop_0^s q\left(\bm {r_0} + s'\bm\Omega,\bm\Omega, E\right)\exp{\left(-\intop_0^s \Sigma_t\left(\bm {r_0} + s''\bm\Omega,E\right)ds''\right)}ds'}
\end{dmath}

Now if $\bm {r_0}$ is on the boundary of the region of interest, the incoming flux $\psi\left(\bm {r_0},\bm \Omega, E\right)$ is known.  This allows the angular flux to be calculated at any point $s$ to be calculated along the characteristic direction $\bm r$.

Up to this point, no approximations have been made in deriving the method of characteristics.  One approximation that is necessary to use MOC for practical problems is to assume some spatial shape for the source term $q$.  The simplest of these is the flat source approximation.  In this approximation, $q$ is assumed to be flat along $\bm r$.  This simplifies equation \ref{e:MOCexact} to the following:

\begin{dmath}\label{e:MOCflatsource}
\psi\left(\bm {r_0} + s\bm\Omega,\bm\Omega,E\right) = \psi\left(\bm {r_0},\bm\Omega,E\right)\exp{\left(-\intop_0^s \Sigma_t\left(\bm {r_0} + s'\bm\Omega,E\right)ds'\right)} + \cdots
\end{dmath}
% Things to talk about:
%   Flat/linear sources

\subsection{Source Iteration}

Using MOC, fixed source transport problems can be solved.  However, in realistic calculations it is rare that the scattering or fission sources are known.  In order to solve the transport equation for problems with an unknown source distribution, the equation can be re-formulated as follows:
\hl{Insert source iteration form of the equation}

Now making some initial guess for \hl{$\phi^-1$} allows us to solve for \hl{$\phi^0$}.  This process can be repeated for all \hl{$\phi^n$} using the previous iterate \hl{variable$\phi^{n-1}$}.  This method is known as Source Iteration (SI).

Source iteration also has a convenient physical interpretation.  Selecting zero as the initial guess means that the first iterate \hl{$\phi^0$} is simply equal to the sum of sources of new neutrons.  Plugging the expression for \hl{$\phi^0$} into the equation for \hl{$\phi^1$}, we find that the \hl{$\phi^1$} is 

\hl{make sure to talk about how ridiculously slowly it can converge.  Find some references for the math if possible}}{}

\subsection{Coarse-Mesh Finite Difference}\label{ss:CMFD}

The source iteration method provides a convenient way of solving transport problems that guarantees a positive, correct solution.  However, because the source iteration can take hundreds of iterations to converge for typical reactor problems, it is important to accelerate the convergence.  The most common way of doing this is the Coarse-Mesh Finite Difference (CMFD) method.  This method involves solving a diffusion problem on a coarse grid to get the average magnitude of the flux in each coarse cell.  This is then used to scale the flux solution on the fine mesh cells.  In order to ensure consistency between the transport solution and the lower order diffusion solution, coupling coefficients are calculated on the boundaries of the coarse mesh cells to correct the leakage between the cells.

\ifdraft{To derive this method, we begin with equation \hl{diffusion equation} and apply the multigroup approximation:
\hl{multi-group diffusion equation}

\begin{equation}
\end{equation}

Integrating this equation over some volume $V$ and dividing by the total volume creates and equation for the average scalar flux $\overline{\phi}$ in that cell.
\hl{discretized equation}

\begin{equation}
\end{equation}

The boundary conditions for each cell are the currents at the interface between two cells:
\hl{boundary condition equation}

\begin{equation}
\end{equation}

This equation can be determined for each cell in the problem to set up a linear system.  This system can then be solved using one of a variety of standard linear solvers.

Because the currents in the matrix are calculated from the diffusion equation, an inconsistency between the diffusion and transport solutions will arise.  To account for this, current correction factors are calculated at the cell boundaries using the fine mesh transport calculations.  These correction factors are calculated as follows:
\hl{dhat equation}

\begin{equation}
\end{equation}

This defines a relationship between the scalar fluxes on each sides of the interface and the current at the interface.  This relationship allows us to correct the diffusion current that is used in the CMFD matrix:\hl{corrected diffusion current}

\begin{equation}
\end{equation}

\subsection{Simplified Pn}

\hl{I can actually probably get rid of this section completely by just referring to the axial solver as $P_3$ instead of $SP_3$}

\subsection{Power Iteration}



\subsection{Collision Probabilities Method}}{}
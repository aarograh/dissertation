\section{Boltzmann Equation}

The Boltzmann equation for neutron transport is shown below: \ifdraft{\todo{Some kind of background or derivation of this?}}{}

\begin{gather}\label{e:boltzmann}
\frac{1}{v} \frac{\partial \psi}{\partial t} + \bm \Omega \cdot \bm \nabla \psi + \Sigma_t\left(x,E,t\right)\psi\left(x,E,\bm \Omega,t\right) = \intop_{4\pi} \Sigma_s\left(x,E' \rightarrow E, \bm \Omega' \rightarrow \bm \Omega, t\right) \psi\left(x,E',\bm \Omega',t\right) d\bm \Omega' + \nonumber\\
\frac{\chi\left(E\right)}{4\pi} \intop \nu \Sigma_f\left(x,E,t\right) \psi\left(x,E,\bm \Omega,t\right) dE + Q\left(x,E,\bm \Omega,t\right)
\end{gather}

For many problems, the time dependence of the Boltzmann equation can be neglected.  Doing do eliminates the first term of the equation above.  In order to ensure neutron balance, this equation is normally formulated as an eigenvalue problem.  The eigenvalue $\lambda = \frac{1}{k}$ allows the two sides of the equation to balance.  The cross-sections can then be adjusted until $\lambda = 1$ is achieved, which is the only meaningful solution for steady-state problems.  The eigenvalue form of the Boltzmann equation is shown below:

\begin{gather}\label{e:ssboltzmann}
\bm \Omega \cdot \bm \nabla \psi + \Sigma_t\left(x,E\right)\psi\left(x,E,\bm \Omega\right) = \intop_{4\pi} \Sigma_s\left(x,E' \rightarrow E, \bm \Omega' \rightarrow \bm \Omega\right) \psi\left(x,E',\bm \Omega'\right) d\bm \Omega' + \nonumber\\
\frac{1}{k}\frac{\chi\left(E\right)}{4\pi} \intop \nu \Sigma_f\left(x,E\right) \psi\left(x,E,\bm \Omega\right) dE + Q\left(x,E,\bm \Omega\right)
\end{gather}

\subsection{Multi-group Approximation}

One important approximation that is commonly made to the transport equation is the multi-group approximation.  To make this approximation, an appropriate energy range for the problem of interest is selected.  This energy range is divided up into $N$ energy groups, with each group going from $E_g$ up to $E_{g-1}$.  For light-water reactor problems, it is common to select 0 eV for $E_N$ and 20 MeV for $E_0$.  Once an appropriate group structure has been selected, equation \ref{e:ssboltzmann} (or \ref{e:boltzmann}\todo{Fix equation \#}) is operated on by 

\begin{equation}
\frac{1}{E_{n-1} - E_n}\intop_{E_n}^{E_{n-1}}\left(\cdot\right)dE
\end{equation}

for each energy group, giving the following result:

\begin{gather}
\bm \Omega \cdot \bm \nabla \psi_g + \Sigma_{t,g}\left(x\right)\psi_g\left(x,\bm \Omega\right) = \intop_{4\pi} \Sigma_{s,g'\rightarrow}\left(x,\bm \Omega' \rightarrow \bm \Omega\right) \psi_{g'}\left(x,\bm \Omega'\right) d\bm \Omega' + \nonumber\\
\frac{1}{k}\frac{\chi_g}{4\pi} \sum_{g'=1}^{G} \nu \Sigma_{f,g'}\left(x,t\right) \psi_{g'}\left(x,\bm \Omega\right) + Q_g\left(x,\bm \Omega\right)\label{e:multigroupboltzmann}
\end{gather}

In equation \ref{e:multigroupboltzmann}, $\psi_g$ denotes the average angular flux for group $g$, defined as

\begin{equation}\label{e:multigroupangflux}
\psi_g\left(x,\bm \Omega\right)=\frac{1}{E_{n-1}-E_n} \intop_{E_n}^{E_{n-1}} \psi\left(x,E,\bm\Omega\right) dE
\end{equation}

The quantities that we want to preserve through the use of the multi-group approximation are the total reaction rates.  To do this, we must define the multi-group cross-sections as follows:

\begin{equation}\label{e:multigroupXS}
\Sigma_x,g \psi_g = \frac{1}{E_{n-1}-E_n} \intop_{E_n}^{E_{n-1}} \psi\left(E\right) \Sigma_x\left(E\right) dE \Rightarrow \Sigma_x,g = \frac{\intop_{E_n}^{E_{n-1}} \psi\left(E\right) \Sigma_x\left(E\right) dE}{\intop_{E_n}^{E_{n-1}} \psi\left(x,E,\bm\Omega\right) dE}
\end{equation}

This definition of the multi-group flux and multi-group cross-sections guarantees that the total reaction rates in each group will be preserved.

\ifdraft{\hl{Talk about the dependence of the XS on the solution?}}{}

\subsection{Spherical Harmonics Approximation}

One of the biggest challenges to solving the transport equation is angle dependence of the scattering cross-sections and angular flux.  To simplify the equation while preserving accuracy of the transport equation, the spherical harmonics approximation is applied to the scattering cross-sections.  This approximation assumes that the angle dependence of the scattering cross-sections can be represented as a series of Legendre polynomials with corresponding coefficients, shown in equation \todo{equation \#}:

\begin{equation}\label{e:sphericalHarmonicsXSExpansion}
\end{equation}

All Legendre polynomials are orthonormal, meaning that each $n$-th polynomial has the following properties:

\begin{subequations}
\begin{equation}
\intop_{-1}^1 P_n\left(x\right) dx =
\end{equation}
\begin{equation}
\intop_{-1}^1 P_n\left(x\right) P_m\left(x\right) dx = \delta_{n,m} = \begin{cases} 1, & n = m \\
0, & n \neq m
\end{cases}
\end{equation}
\end{subequations}

Before proceeding, an important simplification must be made to the scattering source.  The differential scattering cross-section depends on the term $\bm \Omega' \cdot \bm \Omega$, the scalar product between the incoming and outgoing directions.  However, this product is simply the cosine of the angle between the two directions $\mu$.  Using this fact, we can rewrite the scattering source in equation \ref{e:ssboltzmann} as follows:

\begin{equation}
\intop_{4\pi} \Sigma_s\left(x,E' \rightarrow E,\bm \Omega' \rightarrow \bm \Omega\right)\psi\left(x,E',\bm\Omega'\right)d\bm\Omega' = \intop_0^{2\pi} \intop_{-1}^1 \Sigma_s\left(x,E' \rightarrow E,\mu'\right) \psi\left(x,E',\bm\Omega'\right) d\mu' d\gamma'
\end{equation}

\ifdraft{\hl{Look at Larsen's notes to remember the exact mathematical details of this}}{}

\subsection{Diffusion Approximation}

\ifdraft{\hl{All over this, but need to finish up the Pn stuff first}}{}

\subsection{Transport Correction}

\ifdraft{\hl{Not 100\% sure if this even needs to exist at all}}{}

\section{Numerical Methods}

A wide variety of numerical methods exist to solve the transport equation, but all of them can be categorized in one of two ways: stochastic (Monte Carlo) or deterministic methods.  Monte Carlo methods use random numbers to directly sample probability distribution functions based on the material cross-sections.  These methods can produce highly accurate, detailed results and make no assumptions about the materials or geometry of the problem.  However, because of the random nature of the method, there is always some uncertainty in the answer.  For large problems, the number of particles that must be simulated to achieve acceptable levels of uncertainty is often too burdensome \ifdraft{\todo{Need a better word}}{} for practical calculations.

All the work discussed in the following sections makes use of deterministic methods.  There are many different deterministic methods, but only the ones relevant to this work will be discussed in the coming sections.

\ifdraft{\subsection{Method of Characteristics}

One method that is commonly used to solve the transport equation is the Method of Characteristics (MOC).  This method allows the transport equation to be solved along a characteristic direction without any approximations, making it useful for complicated geometries, such as those found in nuclear reactors.  To derive this method, the right-hand side of equation \ref{e:ssboltzmann} is replaced by the variable $q$:

\begin{subequations}
\begin{equation}
\bm \Omega \cdot \bm \nabla \psi + \Sigma_t\left(x,E\right)\psi\left(x,E,\bm \Omega\right) = q\left(x,E.\bm \Omega\right)
\end{equation}
\begin{gather}\label{e:MOCqboltzmann}
q\left(x,E,\bm \Omega\right) = \intop_{4\pi} \Sigma_s\left(x,E' \rightarrow E, \bm \Omega' \rightarrow \bm \Omega\right) \psi\left(x,E',\bm \Omega'\right) d\bm \Omega' +\nonumber\\
\frac{1}{k}\frac{\chi\left(E\right)}{4\pi} \intop \nu \Sigma_f\left(x,E\right) \psi\left(x,E,\bm \Omega\right) dE + Q\left(x,E,\bm \Omega\right)
\end{gather}
\end{subequations}

Now the characteristic direction is introduced:

\begin{equation}
\bm r = \bm {r_0} + s \bm \Omega \Rightarrow \begin{cases} x\left(s\right) = x_0 + s\Omega_x \\ y\left(s\right) = y_0 + s\Omega_y \\ z\left(s\right) = z_0 + s\Omega_z\end{cases}
\end{equation}

Substituting this variable into equation \ref{e:MOCqboltzmann} gives the characteristic form of this equation:

\begin{subequations}
\begin{equation}
\frac{\partial \psi}{\partial s} + \Sigma_t\left(\bm r0_0 + s\bm\Omega,E\right)\psi\left(\bm r0_0 + s\bm\Omega,E,\bm \Omega\right) = q\left(\bm r0_0 + s\bm\Omega,E.\bm \Omega\right)
\end{equation}
\begin{gather}\label{e:characteristicboltzmann}
q\left(\bm r0_0 + s\bm\Omega,E,\bm \Omega\right) = \intop_{4\pi} \Sigma_s\left(\bm {r_0} + s \bm\Omega',E' \rightarrow E, \bm \Omega' \rightarrow \bm \Omega\right) \psi\left(\bm {r_0} + s\bm\Omega',E',\bm \Omega'\right) d\bm \Omega' +\nonumber\\
\frac{1}{k}\frac{\chi\left(E\right)}{4\pi} \intop \nu \Sigma_f\left(\bm r0_0 + s\bm\Omega,E\right) \psi\left(\bm {r_0} + s\bm\Omega,E,\bm \Omega\right) dE + Q\left(\bm {r_0} + s\bm\Omega,E,\bm \Omega\right)
\end{gather}
\end{subequations}

This equation is easily solved using an integrating factor

\begin{equation}
\exp{-\intop_0^s \Sigma_t\left(\bm {r_0} + s'\bm\Omega,E\right)ds'}
\end{equation}

giving the following solution:

\begin{gather}\label{e:MOCexact}
\psi\left(\bm {r_0} + s\bm\Omega,\bm\Omega,E\right) = \psi\left(\bm {r_0},\bm\Omega,E\right)\exp{\left(-\intop_0^s \Sigma_t\left(\bm {r_0} + s'\bm\Omega,E\right)ds'\right)} + \nonumber\\
\intop_0^s q\left(\bm {r_0} + s'\bm\Omega,\bm\Omega, E\right)\exp{\left(-\intop_0^s \Sigma_t\left(\bm {r_0} + s''\bm\Omega,E\right)ds''\right)}ds'
\end{gather}

Now if $\bm {r_0}$ is on the boundary of the region of interest, the incoming flux $\psi\left(\bm {r_0},\bm \Omega, E\right)$ is known.  This allows the angular flux to be calculated at any point $s$ to be calculated along the characteristic direction $\bm r$.

Up to this point, no approximations have been made in deriving the method of characteristics.  One approximation that is necessary to use MOC for practical problems is to assume some spatial shape for the source term $q$.  The simplest of these is the flat source approximation.  In this approximation, $q$ is assumed to be flat along $\bm r$.  This simplifies equation \ref{e:MOCexact} to the following:

\begin{gather}\label{e:MOCflatsource}
\psi\left(\bm {r_0} + s\bm\Omega,\bm\Omega,E\right) = \psi\left(\bm {r_0},\bm\Omega,E\right)\exp{\left(-\intop_0^s \Sigma_t\left(\bm {r_0} + s'\bm\Omega,E\right)ds'\right)} + ...
\end{gather}
% Derive it!
% Things to talk about:
%   Flat/linear sources

\subsection{Source Iteration}

Using MOC, fixed source transport problems can be solved.  However, in realistic calculations it is rare that the scattering or fission sources are known.  In order to solve the transport equation for problems with an unknown source distribution, the equation can be re-formulated as follows:
\hl{Insert source iteration form of the equation}

Now making some initial guess for \hl{$\phi^-1$} allows us to solve for \hl{$\phi^0$}.  This process can be repeated for all \hl{$\phi^n$} using the previous iterate \hl{variable$\phi^{n-1}$}.  This method is known as Source Iteration (SI).

Source iteration also has a convenient physical interpretation.  Selecting zero as the initial guess means that the first iterate \hl{$\phi^0$} is simply equal to the sum of sources of new neutrons.  Plugging the expression for \hl{$\phi^0$} into the equation for \hl{$\phi^1$}, we find that the \hl{$\phi^1$} is 

\hl{make sure to talk about how ridiculously slowly it can converge.  Find some references for the math if possible}}{}

\subsection{Coarse-Mesh Finite Difference}\label{ss:CMFD}

The source iteration method provides a convenient way of solving transport problems that guarantees a positive, correct solution.  However, because the source iteration can take hundreds of iterations to converge for typical reactor problems, it is important to accelerate the convergence.  The most common way of doing this is the Coarse-Mesh Finite Difference (CMFD) method.  This method involves solving a diffusion problem on a coarse grid to get the average magnitude of the flux in each coarse cell.  This is then used to scale the flux solution on the fine mesh cells.  In order to ensure consistency between the transport solution and the lower order diffusion solution, coupling coefficients are calculated on the boundaries of the coarse mesh cells to correct the leakage between the cells.

\ifdraft{To derive this method, we begin with equation \hl{diffusion equation} and apply the multigroup approximation:
\hl{multi-group diffusion equation}

\begin{equation}
\end{equation}

Integrating this equation over some volume $V$ and dividing by the total volume creates and equation for the average scalar flux $\overline{\phi}$ in that cell.
\hl{discretized equation}

\begin{equation}
\end{equation}

The boundary conditions for each cell are the currents at the interface between two cells:
\hl{boundary condition equation}

\begin{equation}
\end{equation}

This equation can be determined for each cell in the problem to set up a linear system.  This system can then be solved using one of a variety of standard linear solvers.

Because the currents in the matrix are calculated from the diffusion equation, an inconsistency between the diffusion and transport solutions will arise.  To account for this, current correction factors are calculated at the cell boundaries using the fine mesh transport calculations.  These correction factors are calculated as follows:
\hl{dhat equation}

\begin{equation}
\end{equation}

This defines a relationship between the scalar fluxes on each sides of the interface and the current at the interface.  This relationship allows us to correct the diffusion current that is used in the CMFD matrix:\hl{corrected diffusion current}

\begin{equation}
\end{equation}

\subsection{Simplified Pn}

\hl{I can actually probably get rid of this section completely by just referring to the axial solver as $P_3$ instead of $SP_3$}

\subsection{Power Iteration}



\subsection{Collision Probabilities Method}}{}
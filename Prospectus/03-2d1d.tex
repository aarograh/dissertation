While MPACT has the capability to solve the Boltzmann transport equation using the 3D Method of Characteristics, doing so is too computationally burdensome to be of practical value.  To obtain accurate solutions to the equation more quickly, a method known as 2D/1D is employed.  This method takes advantage of the geometry in LWRs by exploiting the fact that most of the material heterogeneities (and thus sharp flux gradients) occur only in the radial direction.  Axially, the reactor is typically almost uniform for a given radial point.  This allows the detailed, expensive transport calculations to be done in 2D instead of 3D.  A much faster, lower order calculation can then be done in the axial direction to couple a ``stack'' of 2D planar solutions to obtain 3D flux shapes.  This chapter is devoted to discussing the derivation of the 2D/1D equations, how some of the methods from chapter \ref{chap:transport} are applied to the these equations in MPACT, and some of the approximations and limitations of the 2D/1D method.

\section{Derivation}

\subsection{Radial Equations}

To begin, equation \ref{e:multigroupboltzmann} \hl{or Sn equation?} must be integrated in the $z$-direction over some range $\Delta z_i = z_{i+\frac{1}{2}} - z_{i-\frac{1}{2}}$, giving:

\begin{dmath}
{\Omega_x\frac{\partial \psi_{g,z}}{\partial x} + \Omega_y\frac{\partial \psi_{g,z}}{\partial y} + \frac{\Omega_z}{\Delta z_i}\left(\psi_{g,z_{i+\frac{1}{2}}} - \psi_{g,z_{i-\frac{1}{2}}}\right)} + {\Sigma_{t,g}\left(x,y\right)\psi_{g,z}\left(x,y,\bm\Omega\right)} = {\frac{1}{4\pi}\sum_{g'=1}^{G}\intop_{4\pi}\nu\Sigma_{f,g',z}\left(x,y\right)\psi_{g',z}\left(x,y,\bm\Omega'\right)d\Omega'} + {\frac{1}{k_{eff}}\frac{\chi_{g,z}}{4\pi}\sum_{g'=1}^G\intop_{4\pi} \nu\Sigma_{f,g',z}\left(x,y\right)\psi_{g',z}\left(x,y,\bm\Omega'\right)d\Omega'} + {\frac{Q_{g,z}\left(x,y\right)}{4\pi}}
\end{dmath}

The $z$-component of the streaming can now be moved to the right-hand side of the equation and treated as a source term.

\begin{subequations}
\begin{dmath}
{\Omega_x\frac{\partial \psi_{g,z}}{\partial x} + \Omega_y\frac{\partial \psi_{g,z}}{\partial y}} + {\Sigma_{t,g}\left(x,y\right)\psi_{g,z}\left(x,y,\bm\Omega\right)} = {q_{g,z}\left(x,y\right) + L_{g,z}\left(x,y,\Omega_z\right)}
\end{dmath}
\begin{dmath}
q_{g,z}\left(x,y\right) = {\frac{1}{4\pi}\sum_{g'=1}^{G}\intop_{4\pi}\nu\Sigma_{f,g',z}\left(x,y\right)\psi_{g',z}\left(x,y,\bm\Omega'\right)d\Omega'} + {\frac{1}{k_{eff}}\frac{\chi_{g,z}}{4\pi}\sum_{g'=1}^G\intop_{4\pi} \nu\Sigma_{f,g',z}\left(x,y\right)\psi_{g',z}\left(x,y,\bm\Omega'\right)d\Omega'} + {\frac{Q_{g,z}\left(x,y\right)}{4\pi}}
\end{dmath}
\begin{equation}
L_{g,z}\left(x,y,\Omega_z\right) = \frac{\Omega_z}{\Delta z_i}\left(\psi_{g,z_{i+\frac{1}{2}}} - \psi_{g,z_{i-\frac{1}{2}}}\right)
\end{equation}
\end{subequations}

$L_{g,z}\left(x,y,\Omega_z\right)$ is the axial transverse leakage source term for plane $z$

\subsection{Axial Equations}

\section{Implementation}

\section{Approximations and Limitations}

\subsection{Axial Homogenization}

One assumption made in the 2D/1D scheme is that each of the 2D planes is axially homogeneous.  In many cases this assumption is true.  Other times, materials such as spacer grids, end plugs for fuel rods and control tips, or other structural materials may be homogenized, but do not have a large impact on the results, making the axial homogenization within a plane a reasonable approximation.  However, when strongly absorbing materials (or others which may significantly impact neutron behavior) are partially inserted into an axial plane, homogenizing them axially can produce large errors in the solution.  When this occurs, the best way to remedy these errors is to divide the plane into 2 or more planes in such a way that the problematic material is no longer partially inserted into a plane.

While there is a known workaround for this approximation, it has the drawback of adding extra planes to the 2D/1D scheme, which in turn increases the computational burden of the calculations.  Because of this, other methods to improve the axial homogenization have been a topic of interest for reactor components that commonly cause this problem, such as control rods.  This will be discussed in more detail in \hl{the following chapter.}

\subsection{Axial Transverse Leakage Source}

Another significant approximation in 2D/1D relates to the axial transverse leakage source used by the 2D transport solver.  The 1D axial solve and 3D CMFD acceleration are both done on a pin-homogenized coarse mesh.  This results in the assumption that the axial currents are the same in each part of the pin.  Obviously, this is not true since the fuel and moderator regions will have different energy spectra, thus different currents.  This assumption introduces error in source used in the 2D transport calculations.

\subsection{Radial Transverse Leakage Source}
Moving forward, there are several improvements that can be made to the 2D/1D scheme with regard to the cusping problem.  Four of these improvements will be discussed in this section.  The first two are modifications to methods already in MPACT, while the second two would involve implementation of new solvers.

\subsection{Axial Transverse Leakage Source}

The first improvement that could be made is in the axial transverse leakage source.  As discussed in section \todo{number}, the axial TL is calculated on the homogenized CMFD mesh by the axial solver (usually SP$_3$).  Because of this, it is assumed to be spatially flat within each pin cell.  However, the energy dependence of the flux and current are completely different between the fuel and moderator.  This is also true for a rodded pin cell due to the control rod being a strong absorber.  Thus, the results of the MOC calculations could be improved by treating the spatial dependence of the axial TL source within each pin cell.  \todo{prior work?}

\subsection{Radial Transverse Leakage Source}

A second improvement that could be made is to improve the resolution of the radial TL source when using the sub-plane--based decusping methods.  The current implementation assumes the same $\hat{D}$ current correction term for all sub-plane surfaces between two pin cells.  This assumption is made to improve the stability of the sub-plane scheme, but certainly introduces some error in the CMFD and SP$_3$ calculations.  When a control rod is partially inserted into an MOC plane, this error is greater than normal since the currents in the rodded sub-planes should be much different from those in the unrodded sub-planes.  Introducing a better shape to the $\hat{D}$ terms would certainly improve the overall accuracy of the calculations.

\subsection{Auxiliary Solver}

Another potential source of error for the decusping treatments discussed here is in the choice of auxiliary solver.  The 1D CP kernel was useful for capturing some of the radial cusping effects, but it does have some deficiencies.  First, because it is 1D, it ignores the corner effects of the pin cell by treating the moderator as a ring.  This minimizes the directional dependence of the neutrons, changing the radial flux profile generated by the solver.  Second, the 1D CP solver relies on a buffer region obtained by homogenizing the surrounding pin cells.  As with the moderator region, this buffer region is assumed to be annular and homogeneous.  This affects the source term which drives the problem and the behavior of the neutrons which escape the pin cell of interest.

One way to address this would be to use a highly optimized MOC solver.  Solving only a single pin cell with MOC would have minimal impacts on runtime, but would successfully capture the corner effects and more accurately handle the boundary conditions by directly using the results of the full-plane 2D MOC calculations.  This method should show some improvement over the 1D CP method that is currently used.

\subsection{Sub-Ray MOC}

The final potential improvement that will be discussed here is a new ``sub-ray'' MOC method.  This method is still being developed, but will be tested in MPACT to continue this work.  Sub-ray MOC will be treated like regular 2D MOC in most of the 2D plane.  As a ray is swept, the outgoing angular flux and scalar flux contribution for each region is calculated, along with currents at the pin cell boundaries.  However, upon entering a partially rodded region, the ray will ``split.''  The calculations will be done twice in that region: once using a rodded cross-section and source, then a second time using an unrodded cross-section and source.  This will continue until the rodded and unrodded angular fluxes have converged toward each other, at which point the sub-rays will end and become a single MOC ray again.

This method has a significant advantage over other decusping methods because it directly calculates an improved angular flux, which is the fundamental quantity from which other MPACT results are calculated.  If the angular flux is correct, then all other quantities such as scalar flux, current, and power will be correct as well.  Furthermore, this method could prove to be useful for components besides control rods.  For example, spacer grids, burnable poison inserts, and axial blankets are just three examples of reactor components which usually require additional MOC planes in the 2D/1D method.  However, if sub-ray MOC is used, it should negate the requirement for additional MOC planes, further reducing the computational burden of the calculations.

Several questions must be worked out for the sub-ray MOC method.  The first concerns how to combine the sub-ray angular fluxes.  It is possible that calculating the angular fluxes for each axial level explicitly will allow them to be mixed using the rodded and unrodded volume fractions, but something more advanced may be required.  Additionally, it is not known yet how long the sub-rays must exist before recombining to a single ray.  Each of these questions is discussed in further detail in section \todo{num}, where some results from a 1D MOC code are presented.

\section{1D MOC Results}

Since rod cusping is caused by an overly simplistic homogenization of two materials, a 1D MOC code was developed to analyze the impacts of rod cusping on the angular flux.  Understanding the response of the angular flux to homogenized cross-sections can then be used to develop a more effective decusping method that addresses the angular flux during the calculation instead of applying corrections to the cross-sections or scalar flux after a calculation.  This code is capable of performing both k-eigenvalue and fixed fission source calculations.  This allows the resulting fission source from a calculation to be reused in subsequent calculations to isolate the effects of the cross-section homogenization.

Two begin this analysis, VERA Problem 4 was used as a starting point.  The center row of pins across all three assemblies was modeled in 1D, creating a model that had heterogeneous representations of fuel pins, instrument tubes, and control rods.  The control rod cross-sections were then mixed with the moderator cross-sections to simulate a partially inserted control rod.  \todo{Mention cross-section}

With this model set up, 
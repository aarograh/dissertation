\begin{frame}[t]{Motivation}

\begin{itemize}
  \item The 2D/1D Method solves the transport equation for 3D problems
  \begin{itemize}
    \item Faster than direct 3D transport calculations
    \item More accurate than traditional nodal methods
  \end{itemize}
  \item Control rod cusping can occur in 2D/1D
  \begin{itemize}
    \item Occurs when control rods are partially inserted in a 2D plane
    \item Causes large errors due to volume homogenization
    \item Current decusping methods have trade-off between speed and accuracy
  \end{itemize}
  \item New decusping methods are needed for 2D/1D
  \begin{itemize}
    \item Transport-based methods that are efficient and accurate
    \item New methods can reduce the computing resources required for 2D/1D
    \item Other axial components (spacer grids, burnable poison inserts, axial fuel blankets, etc.) could also be simulated more efficiently using advanced decusping methods
  \end{itemize}
\end{itemize}

\end{frame}

%%%%%%%%%%%%%%%%%%%%%%%%%%%%%%%%%%%%%%%%%%%%%%%%%%%%%%%%%%%%%%%%%%%%%%%%%%%%%%%%%

\begin{frame}[t]{Overview}
  
  \begin{itemize}
    \item Implementation of sub-plane scheme
    \begin{itemize}
      \item Captures sub-plane axial information for each 2D plane
      \item Allows coarsening of axial mesh without runtime increase
    \end{itemize}
    \item New decusping methods
    \begin{itemize}
      \item Sub-plane scheme modified to use heterogeneous rodded/unrodded cross-sections
      \item 1D Collision probabilities introduced to improve sub-plane cross-sections
    \end{itemize}
    \item Proposed work - ``sub-ray'' Method of Characteristics
    \begin{itemize}
      \item Modification to 2D MOC to resolve axial heterogeneities
      \item Should resolve transport effects caused by partially inserted rods
      \item Minimal runtime increase
    \end{itemize}
  \end{itemize}

\end{frame}

%%%%%%%%%%%%%%%%%%%%%%%%%%%%%%%%%%%%%%%%%%%%%%%%%%%%%%%%%%%%%%%%%%%%%%%%%%%%%%%%%
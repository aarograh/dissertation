\begin{frame}[t]{Background}
    
\begin{itemize}
  \item Direct 3D calculations are still usually too slow for most practical 
  calculations
  \item 2D/1D method was developed by researchers at Korea Atomic Energy 
  Research Institute (KAERI) and implemented in DeCART 
  \cite{3DHetWholeCoreTransPlanarMOC,DeCARTTheoryManual,MethodsAndPerformanceOfDecart}
  \begin{itemize}
    \item Decomposes problem into a stack of 2D planes
    \item High-fidelity transport calculations used to solve 2D planes
    \item Fast 1D calculations couple 2D planes together axially
    \item Reactor geometry has most of its heterogeneity in the radial 
    direction, so lower order calculations are acceptable in axial direction
  \end{itemize}
  \item University of Michigan (UM) developed DeCART for awhile, but decided to 
  do a new 2D/1D implementation in MPACT \cite{MPACTTheoryManual}
\end{itemize} 
    
\end{frame}

%%%%%%%%%%%%%%%%%%%%%%%%%%%%%%%%%%%%%%%%%%%%%%%%%%%%%%%%%%%%%%%%%%%%%%%%%%%%%%%%%

\begin{frame}[t]{Background}
  
  \begin{figure}[h]
    \centering
    \includegraphics[width=0.8\textwidth]{../figs/2d1d-subplane.png}
  \end{figure}    
  
\end{frame}

%%%%%%%%%%%%%%%%%%%%%%%%%%%%%%%%%%%%%%%%%%%%%%%%%%%%%%%%%%%%%%%%%%%%%%%%%%%%%%%%%

\begin{frame}[t]{Radial Equations}
    
    \begin{itemize}
      \item Average transport equation axially from $z_{k-\frac{1}{2}}$ to 
      $z_{k+\frac{1}{2}}$
      \item Assume cross-sections are axially constant in region of integration
    \end{itemize}
    \begin{dmath*}
        {\Omega_x\frac{\partial \psi_{g}^Z}{\partial x} + 
        \Omega_y\frac{\partial \psi_{g}^Z}{\partial y}} + 
        {\Sigma_{tr,g}\left(x,y\right)\psi_{g}^Z\left(x,y,\bm\Omega\right)} = 
        {q_{g}^Z\left(x,y,\bm \Omega\right)} + 
        {L_{g}^Z\left(x,y,\Omega_z\right)}
    \end{dmath*}
    \begin{dmath*}
        {q_{g}^Z\left(x,y,\bm \Omega\right)} = 
        {\frac{1}{4\pi}\sum_{g'=1}^{G}\intop_{4\pi}\Sigma_{s,g'\rightarrow 
        g}^Z\left(x,y,\bm\Omega'\cdot\bm\Omega\right)\psi_{g'}^Z\left(x,y,\bm\Omega'\right)d\Omega'}
         + {\frac{1}{k_{eff}}\frac{\chi_{g}^Z}{4\pi}\sum_{g'=1}^G\intop_{4\pi} 
        \nu\Sigma_{f,g'}^Z\left(x,y\right)\psi_{g'}^Z\left(x,y,\bm\Omega'\right)d\Omega'}
         + {\frac{Q_{g}^Z\left(x,y\right)}{4\pi}}
    \end{dmath*}
    \begin{align*}
    L_{g}^Z\left(x,y,\Omega_z\right) &= \frac{\Omega_z}{\Delta z_k}\left(\psi_{g,z_{k-\frac{1}{2}}} - \psi_{g,z_{k+\frac{1}{2}}}\right) \approx \frac{J_{g,z_{k-\frac{1}{2}}} - J_{g,z_{k+\frac{1}{2}}}}{4\pi\Delta z_k}
    \end{align*}
    
\end{frame}

%%%%%%%%%%%%%%%%%%%%%%%%%%%%%%%%%%%%%%%%%%%%%%%%%%%%%%%%%%%%%%%%%%%%%%%%%%%%%%%%%

\begin{frame}[t]{Axial Equations}

\begin{itemize}
  \item Average transport equation over x from $x_{i-\frac{1}{2}}$ to 
  $x_{i+\frac{1}{2}}$ and over y from $y_{j-\frac{1}{2}}$ to $y_{j+\frac{1}{2}}$
  \item Assume cross-sections are radially constant in region of integration
\end{itemize}
\begin{dmath*}
{\Omega_z \frac{\partial \psi_{g}^{XY}}{\partial z}} + 
{\Sigma_{tr,g}^{XY}\left(z\right)\psi_{g}^{XY}\left(z,\bm\Omega\right)} = 
q_{g}^{XY}\left(z,\bm\Omega\right) + 
{L_{g}^{XY}\left(z,\Omega_x,\Omega_y\right)}
\end{dmath*}
\begin{equation*}
L_{g}^{XY}\left(z,\Omega_x,\Omega_y\right) \approx 
{\frac{J_{g,x_{i-\frac{1}{2}},y_j} - 
  J_{g,x_{i+\frac{1}{2}},y_j}}{4\pi\Delta x_i}} +
\frac{J_{g,x_i,y_{j-\frac{1}{2}}} - J_{g,x_i,y_{j+\frac{1}{2}}}}{4\pi\Delta 
y_j}
\end{equation*}

\end{frame}

%%%%%%%%%%%%%%%%%%%%%%%%%%%%%%%%%%%%%%%%%%%%%%%%%%%%%%%%%%%%%%%%%%%%%%%%%%%%%%%%%

\begin{frame}[t]{Calculation Flow}

\begin{columns}
  \begin{column}{0.6\textwidth}
    \begin{itemize}
      \item 3D CMFD \cite{SmithCMFDOrig}
      \begin{itemize}
        \item Determines global flux shape to scale fine mesh solution
        \item Calculates radial currents for 1D axial solver
      \end{itemize}
      \item 1D NEM-SP$_3$ \cite{SPnEquations,finnemann1977RodCuspingOrigMention}
      \begin{itemize}
        \item Calculates improved axial currents for 2D solver
      \end{itemize}
      \item 2D MOC \cite{AskerMOCOrig1972,HalsallMOCOrigCACTUS1980}
      \begin{itemize}
        \item Solves for fine mesh scalar flux
        \item Calculates updated radial currents for CMFD calculation
      \end{itemize}
    \end{itemize}
  \end{column}
\begin{column}{0.4\textwidth}
  \begin{figure}[h]
    \centering
    \resizebox{!}{0.7\textheight}{\begin{tikzpicture}[node distance=2cm]

% Begin
\node (start) [startstop] {Start};
\node (init) [io, right of=start, xshift=2.0cm] {Input, Initialize Solution};

% CMFD
\node (homog) [process, right of=init, xshift=2.0cm] {Homogenize CMFD cross-sections};
\node (calcCouplingCoeffs) [process, below of=homog] {Calculate $\tilde{D}$, $\hat{D}$};
\node (3DCMFD) [process, below of=calcCouplingCoeffs] {3D CMFD Calculation};
\node (CMFDupdate) [process, left of=3DCMFD, xshift=-2.5cm] {Update fission source, matrix};
\node (CMFDconv) [decision, below of=3DCMFD, yshift=-1.5cm] {CMFD Flux, k-eff converged?};
\node (proj) [process, below of=CMFDconv, yshift=-1.5cm] {Scale MOC flux with CMFD flux};

% Nodal
\node (radialTL) [process, left of=proj, xshift=-2.5cm] {Calculate radial transverse leakage source};
\node (sp3) [process, left of=radialTL, xshift=-2.0cm] {1D SP$_3$ Calculation};

% MOC
\node (axialTL) [process, below of=sp3,yshift=-1.5cm] {Calculate fission source, Axial transverse leakage source};
\node (MOC) [process, right of=axialTL, xshift=2.0cm] {Perform 2D MOC sweeps};
\node (MOCconv) [decision, right of=MOC,xshift=2.5cm] {Fission source converged?};
\node (output) [io, below of=MOCconv,yshift=-1.5cm] {Output};
\node (stop) [startstop, below of=output] {Stop};

% Basic Arrows
\draw [arrow] (start) -- (init);
\draw [arrow] (init) -- (homog);
\draw [arrow] (homog) -- (calcCouplingCoeffs);
\draw [arrow] (calcCouplingCoeffs) -| (CMFDupdate);
\draw [arrow] (CMFDupdate) -- (3DCMFD);
\draw [arrow] (3DCMFD) -- (CMFDconv);
\draw [arrow] (CMFDconv) -- node[anchor=west] {yes} (proj);
\draw [arrow] (proj) -- (radialTL);
\draw [arrow] (radialTL) -- (sp3);
\draw [arrow] (sp3) -- (axialTL);
\draw [arrow] (axialTL) -- (MOC);
\draw [arrow] (MOC) -- (MOCconv);
\draw [arrow] (MOCconv) -- node[anchor=west] {yes} (output);
\draw [arrow] (output) -- (stop);

% Fancy Arrows
\draw [arrow] (CMFDconv) -| node[anchor=north] {no} (CMFDupdate);
\draw [arrow] (MOCconv) -| node[anchor=north] {no} ([xshift=0.5cm]CMFDconv.east) |- (homog);

\end{tikzpicture}}
  \end{figure}
\end{column}
\end{columns}

\end{frame}

%%%%%%%%%%%%%%%%%%%%%%%%%%%%%%%%%%%%%%%%%%%%%%%%%%%%%%%%%%%%%%%%%%%%%%%%%%%%%%%%%

\begin{frame}[t]{3D CMFD}
    
    \begin{columns}
      \begin{column}{0.6\textwidth}
        \begin{itemize}
          \item Diffusion-based acceleration
          \item Homogenize fine mesh flux, cross-sections, $\chi$ into coarse 
          mesh cells
          \item Calculate $\hat{D}$ coupling coefficients
          \begin{equation*}\scriptstyle
          \hat{D}_{g,s} = \frac{J_{g,s}^{MOC,k-1} + 
            \tilde{D}_{g,s}\left(\phi_{g,p}^{CMFD,k} - 
            \phi_{g,m}^{CMFD,k}\right)}{\left(\phi_{g,p}^{CMFD,k} + 
            \phi_{g,m}^{CMFD,k}\right)}
          \end{equation*}
          \item Project coarse mesh solution to fine mesh
          \begin{equation*}\scriptstyle
          \phi_{g,j}^{MOC,k} = c_{g,i}^k \phi_{g,j}^{MOC,k-1}
          \end{equation*}
          \begin{equation*}\scriptstyle
          c_{g,i}^k = \frac{\phi_{g,i}^{CMFD,k}}{\phi_{g,i}^{CMFD,k-1}}
          \end{equation*}
        \end{itemize}
      \end{column}
    \begin{column}{0.4\textwidth}
      \begin{figure}[h]
        \centering
        \resizebox{!}{0.7\textheight}{\begin{tikzpicture}[node distance=2cm]

% Start
\node (start) [startstop] {Start};

% CMFD
\node (homog) [process, right of=start, xshift=2.0cm] {Homogenize Cross-sections and Flux; Calculate $\tilde{D}$};
\node (iterCheck) [decision, below of=homog, yshift=-1.5cm] {First Iteration?};
\node (firstIter) [process, below of=iterCheck, xshift=-2.5cm, yshift=-1.0cm] {Set $\hat{D}=0$};
\node (laterIter) [process, below of=iterCheck, xshift=2.5cm, yshift=-1.0cm] {Calculate $\hat{D}$};
\node (matrix) [process, below of=firstIter, xshift=2.5cm] {Set up CMFD Matrix};
\node (3DCMFD) [process, below of=matrix] {3D CMFD Calculation};
\node (proj) [process, below of=3DCMFD] {Scale MOC flux with CMFD flux};

% Stop
\node (stop) [startstop, right of=proj, xshift=2.0cm] {Stop};

% Basic Arrows
\draw [arrow] (start) -- (homog);
\draw [arrow] (homog) -- (iterCheck);
\draw [arrow] (matrix) -- (3DCMFD);
\draw [arrow] (3DCMFD) -- (proj);
\draw [arrow] (proj) -- (stop);

% Fancy Arrows
\draw [arrow] (iterCheck) -| node[anchor=south] {yes} (firstIter);
\draw [arrow] (iterCheck) -| node[anchor=south] {no} (laterIter);
\draw [arrow] (firstIter) |- (matrix);
\draw [arrow] (laterIter) |- (matrix);

\end{tikzpicture}}
      \end{figure}
  \end{column}
  \end{columns}
    
\end{frame}

%%%%%%%%%%%%%%%%%%%%%%%%%%%%%%%%%%%%%%%%%%%%%%%%%%%%%%%%%%%%%%%%%%%%%%%%%%%%%%%%%

\begin{frame}[t]{3D CMFD -- Sub-Plane Scheme}

\begin{columns}
  \begin{column}{0.6\textwidth}
    \begin{itemize}
      \item Split CMFD cells axially into multiple cells
      \begin{itemize}
        \item Calculate shaping factor for each sub-plane cell based on 
        previous solution
        \item Apply shaping factor to fine mesh fluxes during homogenization
        \item $\phi$ and $\chi$ have axial shape; cross-sections are axially 
        constant
      \end{itemize}
      \item Calculate $\hat{D}$ for entire MOC plane; apply to all sub-planes
      \item Modify projection to fine mesh to account for sub-planes
    \end{itemize}
  \end{column}
  \begin{column}{0.4\textwidth}
    \begin{figure}[h]
      \centering
      \resizebox{!}{0.7\textheight}{\begin{tikzpicture}[node distance=2cm]

% Start
\node (start) [startstop] {Start};

% CMFD
\node (homog) [process, right of=start, xshift=2.0cm] {Homogenize Cross-sections and Flux; Calculate $\tilde{D}$};
\node (iterCheck) [decision, below of=homog, yshift=-1.5cm] {First Iteration?};
\node (firstIter) [process, below of=iterCheck, xshift=-2.5cm, yshift=-1.0cm] {Set $\hat{D}=0$};
\node (laterIter) [process, below of=iterCheck, xshift=2.5cm, yshift=-1.0cm] {Calculate $\hat{D}$};
\node (matrix) [process, below of=firstIter, xshift=2.5cm] {Set up CMFD Matrix};
\node (3DCMFD) [process, below of=matrix] {3D CMFD Calculation};
\node (proj) [process, below of=3DCMFD] {Scale MOC flux with CMFD flux};

% Stop
\node (stop) [startstop, right of=proj, xshift=2.0cm] {Stop};

% Basic Arrows
\draw [arrow] (start) -- (homog);
\draw [arrow] (homog) -- (iterCheck);
\draw [arrow] (matrix) -- (3DCMFD);
\draw [arrow] (3DCMFD) -- (proj);
\draw [arrow] (proj) -- (stop);

% Fancy Arrows
\draw [arrow] (iterCheck) -| node[anchor=south] {yes} (firstIter);
\draw [arrow] (iterCheck) -| node[anchor=south] {no} (laterIter);
\draw [arrow] (firstIter) |- (matrix);
\draw [arrow] (laterIter) |- (matrix);

\end{tikzpicture}}
    \end{figure}
  \end{column}
\end{columns}

\end{frame}

%%%%%%%%%%%%%%%%%%%%%%%%%%%%%%%%%%%%%%%%%%%%%%%%%%%%%%%%%%%%%%%%%%%%%%%%%%%%%%%%%

\begin{frame}[t]{1D SP$_3$-NEM}
    
    \begin{columns}
      \begin{column}{0.6\textwidth}
        \begin{itemize}
          \item SP$_3$ \cite{SPnEquations} used to handle angular shape 
          \begin{dmath*}\scriptstyle
            {-\bm{\nabla} \cdot D_{0,g} \left(\bm x\right) \bm \nabla 
            \Phi_{0,g}\left(\bm x\right) + \left[\Sigma_{tr,g}\left(\bm 
            x\right) - \Sigma_{s0,g}\left(\bm 
            x\right)\right]\Phi_{0,g}\left(\bm x\right)} = {Q_g\left(\bm 
            x\right) + 2\left[\Sigma_{tr,g}\left(\bm x\right) - 
            \Sigma_{s0,g}\left(\bm x\right)\right]\Phi_{2,g}\left(\bm 
            x\right)}
          \end{dmath*}
          \begin{dmath*}\scriptstyle
            {-\bm{\nabla} \cdot D_{2,g} \left(\bm x\right) \bm \nabla 
            \Phi_{2,g}\left(\bm x\right) + \left[\Sigma_{tr,g}\left(\bm 
            x\right) - \Sigma_{s2,g}\left(\bm 
            x\right)\right]\Phi_{2,g}\left(\bm x\right)} = 
            {\frac{2}{5}\left\lbrace \left[\Sigma_{tr,g}\left(\bm x\right) - 
            \Sigma_{s0,g}\left(\bm x\right)\right]\left[\Phi_{0,g}\left(\bm 
            x\right) - 2\Phi_{2,g}\left(\bm x\right)\right] - Q_g\left(\bm 
            x\right) \right\rbrace}
          \end{dmath*}
          \item NEM \cite{finnemann1977RodCuspingOrigMention} used to handle spatial shape
          \begin{equation*}\scriptstyle
          Q\left(\xi\right) = \sum_{i=0}^2 q_i P_i\left(\xi\right)\ , \quad 
          \phi\left(\xi\right) = \sum_{i=0}^4 \phi_i P_i\left(\xi\right)
          \end{equation*}
          \begin{equation*}\scriptstyle
          \intop_{-1}^1 P_n\left(\xi\right) 
          \left(-\frac{D}{h^2}\frac{d^2}{d\xi^2}\phi\left(\xi\right) + \Sigma_r 
          \phi\left(\xi\right) - Q\left(\xi\right)\right)d\xi = 0,\ n=0,1,2
          \end{equation*}
          \begin{equation*}\scriptstyle
          \phi_L\left(1\right) = \phi_R\left(-1\right)\ , \quad 
          J_L\left(1\right) = J_R\left(-1\right)
          \end{equation*}
        \end{itemize}
      \end{column}
    \begin{column}{0.4\textwidth}
      \begin{figure}[h]
        \centering
        \resizebox{!}{0.5\textheight}{\begin{tikzpicture}[node distance=2cm]

% Begin
\node (start) [startstop] {Start};

% Nodal
\node (radialTL) [process, right of=start, xshift=2.5cm] {Calculate radial transverse leakage source};
\node (sp3-0) [process, below of=radialTL] {Solve 0th moment equation};
\node (sp3-2) [process, below of=sp3-0] {Solve 2nd moment equation};
\node (convCheck) [decision, below of=sp3-2, yshift=-1.5cm] {Converged?};

% Stop
\node (stop) [startstop, right of=convCheck, xshift=2.5cm] {Stop};

% Basic Arrows
\draw [arrow] (start) -- (radialTL);
\draw [arrow] (radialTL) -- (sp3-0);
\draw [arrow] (sp3-0) -- (sp3-2);
\draw [arrow] (sp3-2) -- (convCheck);
\draw [arrow] (convCheck) -- node[anchor=north] {yes} (stop);

% Fancy Arrows
\draw [arrow] (convCheck) -| node[anchor=north] {no} ([xshift=-1.5cm]sp3-2.west) |- (sp3-0);

\end{tikzpicture}}
      \end{figure}
    \end{column}
    \end{columns}
    
\end{frame}

%%%%%%%%%%%%%%%%%%%%%%%%%%%%%%%%%%%%%%%%%%%%%%%%%%%%%%%%%%%%%%%%%%%%%%%%%%%%%%%%%

\begin{frame}[t]{2D MOC}

\begin{columns}
  \begin{column}{0.6\textwidth}
    \begin{itemize}
      \item Solve along a specific direction $\Omega_n$
      \begin{equation*}\scriptstyle
      \bm r = \bm {r_0} + s \bm \Omega_n \Rightarrow \begin{cases} 
      x\left(s\right) = x_0 + s\Omega_{n,x} \\ y\left(s\right) = y_0 + 
      s\Omega_{n,y} \\ z\left(s\right) = z_0 + s\Omega_{n,z} \end{cases}
      \end{equation*}
      \item Problem reduces from PDE to ODE that can be solved analytically
      \begin{equation*}\scriptstyle
      \frac{\partial \psi_{g,n}}{\partial s} + \Sigma_{t,g}\left(\bm r_0 + 
      s\bm\Omega_n\right)\psi_{g,n}\left(\bm r_0 + s\bm\Omega_n\right) = 
      q_{g,n}\left(\bm r_0 + s\bm\Omega_n\right)
      \end{equation*}
      \begin{dmath*}\scriptstyle
        {\psi_{g,n}\left(\bm r_0 + s\bm\Omega_n\right) = \psi_{g,n}\left(\bm 
        r_0\right)\exp{\left(-\intop_0^s \Sigma_{t,g}\left(\bm r_0 + 
        s'\bm\Omega_n\right)ds'\right)}} + {\intop_0^s q_{g,n}\left(\bm r_0 + 
        s'\bm\Omega_n\right)\exp{\left(-\intop_0^{s'} \Sigma_{t,g}\left(\bm r_0 
        + s''\bm\Omega_n\right)ds''\right)}ds'}
      \end{dmath*}
    \end{itemize}
  \end{column}
  \begin{column}{0.4\textwidth}
      \includegraphics[width=\textwidth]{modular_rays.png}
  \end{column}
\end{columns}

\end{frame} 

%%%%%%%%%%%%%%%%%%%%%%%%%%%%%%%%%%%%%%%%%%%%%%%%%%%%%%%%%%%%%%%%%%%%%%%%%%%%%%%%%

\begin{frame}[t]{2D MOC}

    \begin{itemize}
      \item Assume flat source, cross-section along track with 
      length $L_j$ and spacing $\delta x$
      \begin{columns}
        \begin{column}{0.6\textwidth}
      \begin{align*}\scriptstyle
      \psi^{out}_{g,i,n,j} &\scriptstyle = \psi^{in}_{g,i,n,j}e^{-\Sigma_{t,g,i} 
      L_j} 
      \\\scriptstyle
      &\scriptstyle + \frac{q_{g,i,n}}{\Sigma_{t,g,i}}\left(1 - 
      e^{-\Sigma_{t,g,i}L_j}\right) \\\scriptstyle
      \overline{\psi}_{g,i,n,j} &\scriptstyle = 
      \frac{q_{g,n,i}}{\Sigma_{t,g,i}} 
      \\\scriptstyle
      &\scriptstyle + \frac{1 - e^{-\Sigma_{t,g,i} 
          L_j}}{L_j\Sigma_{t,g,i}}\left(\psi^{in}_{g,i,n,j} - 
      \frac{q_{g,n,i}}{\Sigma_{t,g,i}}\right) \\\scriptstyle
      \overline{\psi}_{g,i,n} &\scriptstyle = \frac{\sum_j 
      \overline{\psi}_{g,i,n,j} \delta x L_j}{\sum_j \delta x L_j}
      \end{align*}
    \end{column}
  \begin{column}{0.4\textwidth}
  \includegraphics[width=\textwidth]{modular_rays.png}
\end{column}
\end{columns}
      \item Modular ray tracing can be used to minimize storage requirements by 
      tracing only portions of problem geometry
    \end{itemize}

\end{frame} 

%%%%%%%%%%%%%%%%%%%%%%%%%%%%%%%%%%%%%%%%%%%%%%%%%%%%%%%%%%%%%%%%%%%%%%%%%%%%%%%%%

\begin{frame}[t]{2D MOC}
  
  \begin{columns}
    \begin{column}{0.55\textwidth}
      \begin{itemize}
        \item Perform ray tracing and store segment information up front
        \item Set up scattering, fission, and axial transverse leakage sources
        \begin{itemize}
            \item Multi-group sweeping
          \item 1-group sweeping
        \end{itemize}
        \item Parallel Decomposition
        \begin{itemize}
          \item Spatial (Planar and Radial)- MPI
          \item Angle - MPI
          \item Ray - OpenMP
        \end{itemize}
      \end{itemize}
    \end{column}
    \begin{column}{0.45\textwidth}
      \begin{figure}[h]
        \centering
        \resizebox{!}{0.7\textheight}{\begin{tikzpicture}[node distance=2cm]

% Begin
\node (start) [startstop] {Start};
\node (init) [io, right of=start, xshift=2.5cm] {Input N$_{inners}$};

% MOC
\node (begin) [process, below of=init] {Set $n=0$};
\node (source) [process, below of=begin] {Calculate fission and axial transverse leakage sources};
\node (scatSource) [process, below of=source] {Calculate scattering source};
\node (MOC) [process, below of=scatSource] {2D MOC sweep for each plane};
\node (MOCdone) [decision, below of=MOC, yshift=-1.5cm] {$n = N_{inners}$?};z
\node (stop) [startstop, right of=MOCdone, xshift=2.5cm] {Stop};

% Basic Arrows
\draw [arrow] (start) -- (init);
\draw [arrow] (init) -- (begin);
\draw [arrow] (begin) -- (source);
\draw [arrow] (source) -- (scatSource);
\draw [arrow] (scatSource) -- (MOC);
\draw [arrow] (MOC) -- (MOCdone);

% Fancy Arrows
\draw [arrow] (MOCdone) -| node[anchor=north] {no} ([xshift=-1.5cm]MOC.west) |- (scatSource);
\draw [arrow] (MOCdone) -- node[anchor=north] {yes} (stop);

\end{tikzpicture}}
      \end{figure}
    \end{column}
  \end{columns}

\end{frame}
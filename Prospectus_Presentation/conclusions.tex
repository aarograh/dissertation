\begin{frame}[t]{Conclusions}
    
    \begin{itemize}
        \item Rod cusping problem explained and demonstrated
        \item New decusping methods developed
        \begin{itemize}
            \item Improved accuracy over previous methods
            \item Minimal runtime increases
            \item Still room for improvement
        \end{itemize}
        \item 1D MOC results showed effects of volume homogenization on angular flux
        \item Sub-ray MOC method development motivated by current decusping methods and 1D MOC results
    \end{itemize}
    
\end{frame}

%%%%%%%%%%%%%%%%%%%%%%%%%%%%%%%%%%%%%%%%%%%%%%%%%%%%%%%%%%%%%%%%%%%%%%%%%%%%%%%%%

\begin{frame}[t]{Next Steps}
    
    \begin{table}[h]
        \centering
        \resizebox{\textwidth}{!}{\begin{tabular}{|c|c|c|}\toprule
            Task & Description & Target Date \\\midrule
            1 & Analysis of cross-section and source effects on angular flux & 10/2016 \\\midrule
            2 & Development of sub-ray MOC method & 12/2016 \\\midrule
            3 & Prototype of method in 1D MOC code & 03/2017 \\\midrule
            4 & Implementation of method in MPACT & 06/2017 \\\midrule
            5 & Testing on VERA Problem 4, 5, 9, and transient test problem & 08/2017 \\\bottomrule
        \end{tabular}}
    \end{table}
    
\end{frame}

%%%%%%%%%%%%%%%%%%%%%%%%%%%%%%%%%%%%%%%%%%%%%%%%%%%%%%%%%%%%%%%%%%%%%%%%%%%%%%%%%

\begin{frame}[t]{Acknowledgments}
    \footnotesize
    \begin{itemize}
      \item This material is based upon work supported under an Integrated University Program Graduate Fellowship.
      \item This research was supported by the Consortium for Advanced Simulation of Light Water Reactors (www.casl.gov), an Energy Innovation Hub (http://www.energy.gov/hubs) for Modeling and Simulation of Nuclear Reactors under U.S. Department of Energy Contract No. DE-AC05-00OR22725.
      \item This research also made use of resources of the Oak Ridge Leadership Computing Facility at the Oak Ridge National Laboratory, which is supported by the Office of Science of the U.S. Department of Energy under Contract No. DE-AC05-00OR22725.
      \item This research made use of the resources of the High Performance Computing Center at Idaho National Laboratory, which is supported by the Office of Nuclear Energy of the U.S. Department of Energy under Contract No. DE-AC07-05ID14517.
    \end{itemize}
    
\end{frame}